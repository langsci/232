\documentclass[output=paper]{langsci/langscibook} 
\author{Iryna Pogorelova\affiliation{Universitat Pompeu Fabra}\lastand{Mireia Trenchs-Parera}\affiliation{Universitat Pompeu Fabra}}
\title{An exploration of life experiences during Study Abroad: A case study of bilingual students and their process of intercultural adaptation}
\shorttitlerunninghead{An exploration of life experiences during Study Abroad}
 
 
\abstract{This qualitative case study investigates intercultural adaptation during a study abroad (SA) of undergraduate bilingual Catalan/Spanish students, focusing specifically on life experiences which resulted in inflection points in that process of adaptation. Data were collected through a pre-departure questionnaire, individual interviews conducted before and after SA, and narrative diaries collected during the sojourn. The results point to the complexity and the nuances of the adaptation process, the diversity of life experiences that may affect adaptation \citep{Bennett1993} and the need to expand existing explanatory theoretical models.  }
\maketitle

\begin{document} 
 
 


\section{Introduction}

Over the last two decades, European policies aiming at promoting European citizenship have reduced border-crossing formalities for EU citizens with legislation reforms facilitating and promoting free mobility. To prepare undergraduate students for such a common space, study abroad (SA) programmes have become more accessible, and an increasing number of universities establish a compulsory stay abroad in the curriculum. The logic for higher education administrators behind such policy is clear: the SA experience, even of short duration, may help students not only improve their \isi{foreign language} ability, but also develop their \isi{intercultural} competence (IC), namely, knowledge, skills and attitudes needed to communicate effectively (\citealt{Byram1997,Deardorff2006}) and behave appropriately in a foreign context. 

Academic mobility has been on the rise in Europe at the same time as increasing numbers of sojourners grow up in diverse \isi{linguistic} and cultural backgrounds. They are often bilingual, trilingual or even \isi{multilingual} and are used to sharing globalized classrooms and neighbourhoods with people from other cultures (\citealt{Trenchs-PareraNewman2015}). While abroad, they are supposed to adjust to a different culture and to speak a \isi{foreign language}, both within and outside the host university, and use it for demanding academic purposes. 

Academic and social interest in the consequences of this growing phenomenon has triggered much research in the last decades (for reviews, see \citealt{Williams2005,BehrndPorzelt2012}). Much of it focuses on \isi{L2} \isi{acquisition} (\citealt{MitchellEtAl2015}) since SA experience has been seen as an opportunity for \isi{immersion} in the target culture and, thus, a key for \isi{linguistic} improvement. Research has been predominantly, though not exclusively, quantitative and it has scarcely focused on bilinguals who come from multicultural and \isi{multilingual} backgrounds (but see \citealt{Pérez-Vidal2014}). Although there have been longitudinal studies investigating personal and academic experiences of students (for instance, \citealt{Isabelli-García2006}), few have explored how such experiences relate to students’ adaptation to a new culture and their development of \isi{intercultural} competence (but see \citealt{Beaven2012} and \citealt{VandeBergEtAl2009}). 

This longitudinal multi-case study seeks to shed further light on personal and academic experiences during SAs by tapping into how a group of bilingual undergraduate students from a multicultural and \isi{multilingual} society adapt to a new environment during a short- or mid-term SA programme.   We specifically focus on those life experiences that influence the most their adaptation to the host country \cite{Bennett1993}. In this study, such experiences —whether having a positive or negative impact— are therefore either positive or negative “\isi{inflection} points” in their expected process of adaptation to the host culture.

We were interested in focusing on bilingual sojourners to explore the common assumption that they are more responsive to and faster in adapting to a new country just because they are used to switching from one language to another in their everyday lives (\citealt{Bialystok2001,VedderVirta2005}). Furthermore, we wanted to choose sojourners that, besides being bilinguals, had lived in a context as \isi{multilingual} and multicultural as possible and, consequently, had already gone through \isi{intercultural} experiences before their SA. Present-day Catalonia seemed to be an ideal context because of its high immigrant population and its historical, institutional bilingualism – \ili{Catalan} and \ili{Spanish} have been co-official since 1983 and are spoken everywhere whereas Aranese Occitan, Catalonia’s third co-official language, is of preferential institutional use in the Pyrenean Valley of Aran (\citealt{NewmanEtAl2008,Trenchs-PareraNewman2015}).


\section{Literature review: intercultural adaptation and study abroad}\label{sec:pogorelova:2}

Various scholars have used different terminology to define people’s reactions to \isi{intercultural} contacts (\citealt{Berry1997,Ward2004,MasgoretWard2006}). The concepts such as adjustment, adaptation, \isi{acculturation}, assimilation, and integration are often used to describe changes in how people respond to cultural differences and act in a new milieu. \citet{Kim2005} refers to adaptation as a broad concept that includes all the above-mentioned terms. She defines the process itself as “the entirety of the phenomenon of individuals who, upon relocating to an unfamiliar \isi{sociocultural} environment, strive to establish and maintain a relatively stable, reciprocal and functional relationship with the environment” (p.380). Whatever term is adopted, change (for instance, changes in attitudes, motivations or beliefs) and personal transformation are two core constituents of the process of “fitting” into a new social and cultural environment.

In an attempt to explain how people react to changes and behave in a new cultural setting, some theoretical models have emerged. Synthetically, the U-curve hypothesis model \citep{Lysgaard1955} points at the existence of a period of excitement, followed by \isi{culture shock} and a final recovery from the crisis. The W-curve model (\citealt{GullahornGullahorn1963}) adds a possible shock upon return. The culture-learning model of cross-cultural adjustment (\citealt{Ward2004,MasgoretWard2006}) posits that sojourners adapt well if they acquire culture-specific skills, including the knowledge of the language spoken in the host culture; in this case, the model is represented with an upward learning curve. \citet{Ward2004} further suggests that adjustment operates on two different levels, psychological and \isi{sociocultural}, which are predicted by different factors. Hence, such factors as personality, expectations and social support might have an influence on psychological adjustment, whereas cultural knowledge, cultural distance, previous experience abroad, language \isi{fluency}, length of stay abroad, amount of contact with host nationals, and the quality of this contact might affect \isi{sociocultural} adjustment. The integrative model proposed by \citet{Kim2005} focuses on a deculturation-\isi{acculturation} process depicted in a form of a spiral that develops in an upward direction in a “draw-back-to-leap” pattern. Finally, \citegen{Bennett1993} Developmental Model of Intercultural Sensitivity (DMIS) describes adaptation as a stage towards effective \isi{intercultural} communication and ethnorelativism. Although this model is a general framework to describe people’s reactions to cultural differences and does not address mobility, it may appropriately be applied to a SA context since it posits that adaptation entails the \isi{acquisition} of necessary skills for \isi{intercultural} communication and the final display of either empathy or cultural pluralism.

The aforementioned models explaining adaptation in a variety of contexts have been used in empirical studies that explore \isi{intercultural} adaptation. However, they have been put largely into question and continue to be debated.  With a longitudinal perspective, \citet{Beaven2012} carried out qualitative research to find out the effect of Erasmus mobility on \ili{Italian} university students’ adaptation. Having explored students’ personal and academic life experiences, the scholar concludes that adaptation is not the linear process most early models \cite{Bennett1993}, such as students’ motivation and expectations, personality traits, coping strategies, social and \isi{linguistic} abilities, and stay abroad conditions. 

Some studies have explored students’ motivation as a crucial factor in the culture-learning process. Sojourners’ attitudes towards the host society, openness to other cultures, and willingness to take part in social communication with host nationals contribute to what \citet{Gardner1985} calls “integrativeness.” Accordingly, the level of integrativeness is considered to be a predictor for host language \isi{acquisition}, which in turn has influence on the amount of interaction with hosts and is correlated with a decrease in social adaptation difficulties \citep{Ward2004}. More recently, \citet{BadstübnerEcke2009} have acknowledged the importance of considering, not just the amount of interaction with host nationals, but also the quality and depth of those interactions. 

A number of studies have attempted to correlate students’ expectations regarding SA experience with their adjustment. Some scholars report that sojourners whose experiences exceed initial expectations demonstrate a higher level of adaptation than those whose pre-departure expectations are unfulfilled \citep{Ward2004}. Apart from expectations, research also highlights the role of social support during cross-cultural transitions. \citegen{Pitts2009} ethnographic study revealed that everyday talk with co-nationals helped students reconsider their expectations, which in turn contributed to their gradual adjustment and to the development of an \isi{intercultural} identity. Friendships with co-nationals during SA, however, are not always reported as being a facilitator of adaptation. Instead, the results of \citegen{HendricksonEtAl2011} survey-based study align with the studies that emphasize the importance of building a network of friends from the host country during the sojourn. The implications of their study highlight the importance of receiving \isi{intercultural} and social support training before departure, finding housing with locals and using the classroom as a platform for building friendships with hosts.  

SA research informs, though, about other facilitators of adaptation. Thus, Van\-de Berg and his colleagues (\citeyear{VandeBergEtAl2009}) investigated the extent to which SA contributed to American university students’ \isi{intercultural} learning and host language \isi{acquisition}. Their findings indicated that prior language study in high schools and colleges, the length of the mobility programme, blended class composition and the amount of time spent with host nationals were the best predictors of their students’ \isi{intercultural} development.  Some scholars report that sojourners who had studied or had lived abroad demonstrated a higher degree of adjustment in further \isi{intercultural} encounters \citep{Ward2004}, while other scholars cannot confirm this correlation (\citealt{VandeBergEtAl2009}). Finally, an increasing number of studies have found that SA experience alone does not guarantee the improvement of \isi{intercultural} communication skills. Pre-departure \isi{intercultural} training and cultural mentoring during SA are shown to play an important role for students’ awareness, as well as for adjustment (\citealt{Williams2005,Jackson2008a,VandeBergEtAl2009}).

Most existing research into such issues is quantitative with a pre- and post-test group design. On the contrary, few studies (\citeauthor{Jackson2008b} \citeyear*{Jackson2008b}; \citealt{BownEtAl2015,Campbell2015,Plews2015}) involve the analysis of students’ narratives of daily experiences both during and after the sojourn. At the same time, little is still known about the effect of SA on \isi{intercultural} adaptation of bilingual sojourners accustomed to \isi{multilingual} and multicultural backgrounds at home, even though bilingual and \isi{multilingual} citizens have become the norm rather than the exception in today’s Europe. To contribute to these gaps, \citet{Pogorelova2016} conducted a longitudinal study combining quantitative and qualitative methods. By collecting participants’ self-assessment of experiences during a SA, she was able to construct graphs with curves representing their individual adaptation processes for both their personal and academic lives abroad. These representations resembled trends described in two previous models, the U- and the learning-shaped curves, but the individual curves displayed fluctuations of various intensities, suggesting that participants’ adaptation process was not always smooth. In fact, some trajectories differed from those models to a large extent and pointed at the need to further explore possible explanations for \isi{inflection} points in their adaptation curves, which is what the present study intends to do.  


\section{Research questions}

The current study seeks to extend research on the effect of SA university programmes on bilingual students’ \isi{intercultural} adaptation by answering the following questions: 

\begin{itemize}
\item[RQ1.] What life experiences during study abroad have an impact on bilingual university students’ \isi{intercultural} adaptation to the country of stay?
\item[RQ2.] To what extent are those life experiences similar or different to those identified in previous literature in relation to monolingual students on SA?
\end{itemize}

\section{Method}


\subsection{Design}


This study took the form of a longitudinal, qualitative, multi-case study with data collected during an academic year from a group of 12 undergraduate students from a university in Catalonia (Spain). We chose this context because we wanted to collect data from a group of bilingual students whose previous schooling had taken place in a de facto \isi{multilingual} and multicultural society, as Catalonia actually is due to recent international immigration \citep{Trenchs-PareraNewman2015}. This university also had its own \isi{language policy} which preserved the use of three working languages —\ili{Catalan}, \ili{English} and \ili{Spanish}— both for administrative and academic purposes in all graduate and undergraduate degrees; therefore, our participants would go on an SA after having experienced academic life in more than one language. The study, as further explained below, includes the use of three different instruments — a profile questionnaire, two interviews and a series of narrative diaries for each participant— used at three different times: before, during and after the participants’ sojourn abroad.


 
\subsection{Participants}
\largerpage

The participants of this study were 12 undergraduate students, 11 females and 1 male, with an age range between 20 and 25 (M=22.3). They were volunteers from four faculties—Health and Life Sciences, Political and Social Sciences, Humanities and Communication—and were enrolled in a non-compulsory SA programme. Their host destinations spread over Europe, North-America, South-America, Chi\-na, and Australia. The length of the sojourns varied from one and a half to six months. To maintain confidentiality, all the participants were offered to adopt the pseudo\-nyms used in the present study.  \tabref{tab:pogorelova:1} summarizes the participants’ profile data.

\begin{table}
\small
\begin{tabularx}{\textwidth}{l@{~}l@{~}rp{7mm}Ql@{~}Y}
\lsptoprule
\bfseries Participant & \bfseries Gender & \bfseries Age & \bfseries Acad year & \bfseries Faculty & \bfseries Destination & \bfseries Length of SA (months)\\
\midrule
Maria & f & 23 & 4th & Health and Life Sciences & Germany & 4\\
Cristina & f & 23 & 4th & Health and Life Sciences & Argentina & 4\\
Virginia & f & 21 & 2nd & Political and Social Sciences & Canada & 4\\
Daniel & m & 25 & 3rd & Political and Social Sciences & USA & 1,5\\
Sara & f & 22 & 2nd & Humanities & Australia & 6\\
Elizabeth & f & 22 & 3rd & Humanities & Brazil & 6\\
Angela & f & 26 & 3rd & Humanities & China & 6\\
Anna & f & 22 & 3rd & Humanities & USA & 6\\
Kira & f & 22 & 3rd & Humanities & Netherlands & 6\\
Lola & f & 21 & 2nd & Communication & Netherlands & 6\\
Ares & f & 20 & 2nd & Communication & Netherlands & 6\\
Natalia & f & 21 & 3rd & Communication & China & 4\\
\lspbottomrule
\end{tabularx} 
\caption{\textit{Participants’} \textit{profile} \textit{data}}
\label{tab:pogorelova:1}
\end{table}

Regarding the participants’ \isi{linguistic} background, all of them were \ili{Catalan}/\linebreak\ili{Spanish} bilinguals and had attended compulsory education in Catalonia. Ten of them had previously attended \isi{foreign language} courses abroad which did not exceed two months. Besides, all had made short sojourns abroad involving school trips, work and summer camps, and tourism, but not with academic purposes. 

In order to participate in the SA programme, the students had to meet a number of \isi{foreign language} requirements. Nine out of twelve participants selected universities where \ili{English} was the language of instruction and their \isi{linguistic} abilities varied between B2 and C1 levels of the CEFR. The other two students (Virginia and Elizabeth) enrolled on courses taught in a \isi{foreign language} other than \ili{English}. Thus, Virginia attended her courses in French, Elizabeth in \ili{Portuguese}. One student, Cristina, did medical practices in \ili{Spanish}-speaking Argentina and did not need any \isi{foreign language} accreditation.  


 
\subsection{Data collection procedures}


The current study adopted a longitudinal design with data collected at three different times —before, during and after the participants’ sojourn abroad. First, volunteers were asked to fill in a profile questionnaire and send it back by e-mail. Second, we organized an informative, individual session to get to know each student personally, to explain the procedures adopted in the study, to inform them that their participation was voluntary and anonymous, and to conduct a pre-departure interview. To collect data during the SA, four virtual Moodle classrooms were created for each of the faculties. In these virtual classrooms, every two weeks, the participants submitted a narrative diary entry. After returning from the SA, another interview was conducted with each of the participants.  

When participating in the interviews and completing the narrative diaries, the participants were free to use the three work languages at the university. However, during the interviews, the great majority expressed the desire to practice their \isi{foreign language} and communication between the interviewer and the participants took place in \ili{English}. 


 
\subsection{Instruments}


The pre-SA individual profile questionnaire was adapted from the \isi{linguistic} profile survey used in the SALA research project\footnote{IRIS is an open-access digital repository \citep{MarsdenEtAl2016} accessible at http://www.iris-database.org).}. The adapted version comprises 16 questions investigating the participants’ cultural and \isi{linguistic} background, and includes questions inquiring into prior SA experiences, frequency of interactions with Erasmus exchange students or language tandems at their home university, and estimated \isi{target language} \isi{proficiency}.

Two semi-structured interviews were conducted with each of the participants before departure and upon return (\citealt{Isabelli-García2006}). The pre-departure interview aimed at completing the data provided in the profile questionnaire, while collecting further data on students’ motives for going abroad, destination choice, and expectations. Moreover, each participant was asked a number of personalized questions depending on his or her answers to the profile questionnaire. The post-SA interview aimed at collecting reflections on their adaptation to the new cultural and academic environment and on life experiences which might have influenced that adjustment. 

During the SA, the participants completed narrative diaries biweekly, adapted from the ones used by \citet{Beaven2012}. First, students were asked to self-evaluate their personal and academic lives on a Likert scale from 1 (very bad) to 5 (very good) according to the degree of difficulty they had experienced during the corresponding two-week period. As in previous literature, responses to these Likert scales would be used to graphically represent adaptation. After that, they were required to leave explanatory comments on their self-assessment. Thus, these narratives served to provide an insight into what experiences influenced their self-assessment during their stays. 

The narratives were guided by ten thematic strands, five of them were adopted from \citegen{Beaven2012} weekly diary tables, and the other five were motivated by previous research on the effect of personal characteristics, SA conditions and social networks (\citealt{VandeBergEtAl2009,Pitts2009,Coleman2015}). We separated the ten thematic strands into two broad categories –Personal Life and Academic Life– also used by Beaven since, according to her results, the personal and \isi{academic experience} may evolve differently. The Personal Life category included six subcategories: (a) relationships with native friends; (b) relationships with friends of other nationalities; (c) relationships with host nationals; (d) daily life; (e) \isi{foreign language} for social interaction; and (f) culture, custom and habits of the host country. In turn, the Academic Life category involved four subcategories: (a) educational system; (b) classes (teachers, classmates, etc.); (c) \isi{foreign language} for academic purposes; and (d) academic support for administrative issues. We are aware that other classifications might have yielded slightly different results. To counterbalance this limitation, we introduced face-to-face interviews which could open the possibility for the emergence of other relevant issues in the students’ lives abroad.


 
\subsection{Data analysis}
 

As in previous literature (see \sectref{sec:pogorelova:2}), we wanted to construct graphs representing individual adaptation curves. To do so, we averaged the mean values of the students’ \isi{intercultural} adaptation biweekly self-assessments. For every student, the mean of his or her responses on a five-point Likert scale was calculated separately for every two-week period and per thematic strand. Upon those mean values, we constructed a set of ten graphs representing ten single adaptation curves for each participant, one for each thematic strand. In these graphs, we set out to detect \isi{inflection} points in the curves and, therefore, in their stay abroad: that is, either a negative change (a “shock”, “difficulty” or “crisis” in the previous explanatory model seeing \isi{intercultural} adaptation as a U-curve, e.g. \citealt{Lysgaard1955}) or a positive change (also called previously a moment of “excitement”) in these self-assessments of \isi{intercultural} adaptation. As \isi{inflection} points, we considered only curve variations that were higher with respect to a previous value, that is, in the curve displaying, e.g. +1 followed by +3, the latter indicator was counted as a positive \isi{inflection}. Negative inflections were counted in the same way.  

Since our main intended focus was to find out possible explanations for those positive and negative \isi{inflection} points, we turned to the participants’ written and oral narratives obtained from the narrative diaries and the interviews. These narratives were subjected to an in-depth thematic analysis \citep{Lichtman2012}. We employed a directed approach to content analysis in which initial coding is guided by relevant theories and previous research findings. With this approach, excerpts in interview transcripts or diary entries appearing either to transmit affective reactions to life experiences or to include factors influencing adjustment were assigned a predetermined category from previous studies (\citealt{Ward2004,Williams2005,VandeBergEtAl2009,Beaven2012}). Any excerpt that did not fall into initial coding categories was assigned a new code. This thematic recursive analysis, similar to \citegen{Beaven2012}, allowed us to identify issues triggering \isi{inflection} points in students’ stays. 

In the following section, we explore the categories relating to personal and academic life experiences that emerged from the data and illustrate them with excerpts from the participants’ comments. The quotes that were originally produced in \ili{Catalan} and \ili{Spanish} have been translated into \ili{English}, but the original versions are left in footnotes. Whenever no \ili{Catalan} or \ili{Spanish} version is presented, the comments were originally provided in \ili{English}. Original punctuation and orthography have been preserved. 


\section{Results}


\subsection{Personal life}


In this section (see \tabref{tab:pogorelova:2}), we present the main themes related to \isi{inflection} points in the students’ adaptation curves for their Personal Life. \tabref{tab:pogorelova:2} below includes the number of participants for whom a life experience related to a given personal issue resulted in either a positive or negative \isi{inflection} point in their stays.

\begin{table}[t]
\begin{tabularx}{\textwidth}{Qr}
\lsptoprule

 \bfseries Personal issues & \bfseries \# of participants (n=12)\\
 \midrule 
Emotional support from family and friends from home & 7\\
\tablevspace 
Relationships with friends prior to the SA over distance & 2\\
\tablevspace 
Social and emotional support from co-nationals & 9\\
\tablevspace 
Making new friendships with international students & 7\\
\tablevspace 
Building friendships with host nationals & 8\\
\tablevspace 
Difficulties in interacting in the \isi{foreign language} either with hosts or internationals & 8\\
\tablevspace 
Feeling improvement in the \isi{foreign language} & 7\\
\tablevspace 
Leisure activities in gatherings and trips & 11\\
\tablevspace 
Combining studies with work outside the university & 2\\
\tablevspace 
Adapting to a new social environment (host transport system, timetable, sanitary conditions, accommodation, food, weather etc.) & 9\\
\lspbottomrule
\end{tabularx} 
\caption{Main themes relating to inflection points in participants’ personal life}
\label{tab:pogorelova:2}
\end{table}

Friends from home and family provided what \citet{Beaven2012} calls “emotional support.” Communication with them was made mostly through Skype and social networks. Nevertheless, relatives’ and friends’ visits were not uncommon either and influenced the participants’ biweekly assessments positively. A few students, however, found it challenging to keep in touch with friends left home and noted that distance had had an impact on their relationships, and that they did not feel their best friends at their side when they needed to share SA experiences (\citealt{ColemanChafer2010}). Apart from friends back home, a great majority maintained close relationships with \ili{Spanish} students from their home university and other co-nationals who were in the same location. These friendships were considered to be helpful without being overbearing and were related to significant, positive \isi{inflection} points in their adaptation curves, as in the case of Angela, who arrived in Beijing and, from the very beginning, made friends with two other \ili{Catalan} girls who stayed in the same residence hall. In her first narrative, she wrote: “Luckily we are becoming more than exchange students mates. We are building something like a real friendship. We are always checking upon each other but we are also letting ourselves private moments to do some things on our own.” Angela’s data showed clearly such positive impact several times throughout the sojourn, coinciding with the moment when she felt overwhelmed by her university studies and work; she was very grateful to her friends for being at her side:

\begin{quote}
Narrative diary (at week 8)
\smallskip\\
This month has been quite hard, to be honest. I went to job interviews, and I've got one. We have the exams very soon and I've been up and down in this huge city, taking the underground every day for more than three hours. I arrived exhausted at night, but I was glad to know that I could talk to the girls, disconnect with them even if it lasted only a moment.
\end{quote}

There were also participants who hardly kept in touch with co-nationals during their SA. For Sara, for instance, her relationships with \ili{Spanish} students met in the host country featured mostly the so-called instrumental friendship. These friends were approachable occasionally in terms of basic needs and functional problems (like to help each other to send a suitcase or get university papers) but, as Virginia did in Quebec, some of our participants tried to set themselves apart from co-nationals during their sojourn. Departing for Canada, Virginia aimed at improving her French and meeting as many host and exchange students as possible. Therefore, she was not interested in interacting with \ili{Spanish} students, which was reported in her narrative diary at week 2: “I try not to interact too much with \ili{Spanish} people because that is certainly NOT what I look for in an exchange program.” 

Besides the co-nationals, the participants spent their time with other international students. Orientation sessions and events organized by the host university’s mobility office or Erasmus Students Network (ESN) were the starting point for building a network of international friends. After having made first acquaintances, the students usually joined a group on Facebook or WhatsApp so as to coordinate joint events or just stay in touch. When compared to a decade ago, this is a fairly recent phenomenon in SAs that allows for more common shared spaces of interaction, albeit virtual, with acquaintances from other countries (\citealt{ColemanChafer2010}). Most of such friendships were made in residence halls and shared apartments. The majority assessed their relationships with other exchange students positively and considered these friendships as being beneficial for their \isi{foreign language} and cultural learning. The main difficulty reported here was the language barrier which made some students feel embarrassed or even refrain from interacting, especially in the initial period.    

As regards host nationals, only a small number of students managed to build meaningful friendships with them throughout the sojourn. In fact, the greater part of the participants experienced difficulties in this respect. The reasons mentioned were \isi{linguistic} difficulties, reluctance on the part of hosts to build friendships, and the Erasmus context itself. Angela, for instance, chose to limit her relationships with \ili{Chinese} at the university almost from the very beginning, as, from her viewpoint, they were only interested in practising their \ili{English}. After university hours, she however communicated more closely with her \ili{Chinese} colleagues at the school where she had a part-time job. Although she was well-received at work, Angela considered these relationships with co-workers superficial as they were not interested in building a friendship, but rather wanted to take advantage of the opportunity to talk in \ili{English} with a foreigner. Beyond the university, Natalia managed to meet some local people in Hong Kong thanks to friends, but the contacts with host nationals were not frequent. The language barrier also complicated communication as, contrary to her expectations, very few local people spoke \ili{English} and she could only speak a bit of \ili{Mandarin} and no \ili{Cantonese}. Sara was one of the students who pointed to the difficulty of meeting hosts in the Erasmus international environment. Most of her time, she was surrounded by people from other countries either at the university or in the apartment where she stayed: 

\begin{quote}
Narrative diary (at week 8)
\smallskip\\
I don't know, here in Australia, it is quite hard to meet people from here being on the Erasmus programme. Most of them are international. I have a bit of relationship with two Australians, but not like very deep relationship. I'm basically with people from all over the world but not from here.  
\end{quote}

Assessing their relationships either with hosts or internationals, students verbalised \isi{linguistic} difficulties in their interactions although none had anticipated significant problems communicating with other students  before departure. Such beliefs were underpinned mainly by the students’ prior experiences abroad. As can be judged by their comments, the recognition of their own \isi{linguistic} limitations on-campus and off-campus came later during the sojourn, which then caused disappointment and sometimes even led to what \citet{Beaven2012} calls “\isi{foreign language} exhaustion.” In these moments of fatigue, some students were willing to find a refuge among their co-national friends with whom they could speak \ili{Spanish} or \ili{Catalan}, and there was no need to make an effort explaining everything in a \isi{foreign language}. Lack of \isi{fluency}, vocabulary limitations, the difficulty of understanding jokes and a strong local \isi{accent} were among the reasons for negative \isi{inflection} points. In her narrative diary at week 4, Sara verbalizes her feelings as “. . . many times I do not understand what they're talking about ... and it's a little bit frustrating.” At the end of the SA, however, most felt that their overall \isi{foreign language} had been considerably enhanced and were satisfied with the experience that allowed them to practise the \isi{target language} in different contexts. 

One word also needs to be said about Cristina’s experience in \ili{Spanish}-speak\-ing Argentina. Despite being a \isi{native speaker} of \ili{Spanish}, she faced \isi{linguistic} problems when trying to make friends with hosts at the start of her sojourn. She reported that the use of some words in Argentina was considerably different from their use in Barcelona and she needed to select words carefully:

\begin{quote}
Narrative diary (at week 2)
\smallskip\\
 In spite of the same language during the first days I did not understand a lot of words. And the most important thing is the forbidden word "coger," we use it to say to catch a flight or to catch something ... and there it means going to bed with someone, so I have to constantly think about what I have to say in order not to use completely different words to say the same things.\footnote{Original quote in \ili{Spanish}: “Todo y ser el mismo idioma los primeros días no entendía muchas palabras. Y lo más importante la palabra prohibida “coger”, para nosotros es coger un vuelo, coger algo… y ahí significa ir a la cama con alguien, así que he de pensar constantemente que he de decir por no hablar de palabras completamente diferentes para decir las mismas cosas.”}
\end{quote}

In the participants’ daily life, tourism and leisure activities triggered a great deal of positive emotions. The great majority travelled around the area or even outside the host country at the weekends, on days off and on holiday. These experiences allowed them to leave their international on-campus environment and to see the host country in a new perspective, as well as to explore other cultures beyond its borders. For instance, Ares, who was studying in Groningen, explored her host city and also visited Berlin with her new international friends during a week-off. Natalia, who studied in Hong Kong, visited other \ili{Chinese} provinces. She also had the opportunity to visit other Asian countries, such as Malaysia, Japan, and Korea. Daniel, who stayed in Los Angeles, made a five-day road trip with his \ili{Catalan} friends in the States. Virginia, who stayed in Quebec, made a road trip with a group of her new French friends around Canada. Later at the end of her exchange programme, she also went to Mexico with the same people. All these experiences translated into positive \isi{inflection} points in their adaptation curves. 

Accommodation, food and weather were sometimes among the reasons that affected negatively their stay, at least initially. Some students, for instance, had to get used to sharing showers and kitchens in residence halls and others lived in small apartments which made them miss their room back home. However, these hurdles did not incite them to leave their SA programmes and, in the end, some participants even reported that such shared facilities made them feel integrated into the community and contributed to their adaptation to customs and habits from other countries. Besides, the experience of living alone was considered enriching in terms of personal growth, as it made the majority feel like real adults. Completing errands and chores on their own, such as going shopping, cooking, or doing the laundry, made them feel more independent and self-sufficient. 

A few students, who were working during their residence, had difficulty balancing work commitments and university studies, and this combination resulted in stress, especially in the period of exams. Nevertheless, the overall assessment of working experience was positive as these participants reported that working had contributed to their maturity and independence. Sara’s quote illustrates the positiveness of such personal experiences that seems to trigger a sense of self-discovery and feeling of readiness to confront new cultural challenges:

\begin{quote}
Narrative diary (at week 12) 
\smallskip\\
Super good, feeling of total freedom! I had never lived away from home and had never earned my own salary, so explosive mixture of freedom! [...] I feel much more mature, independent, sure of myself, with more strengths. 
\end{quote}

Interestingly, many found their host culture to be similar to their own and felt they did not need to step outside of their comfort zone. However, it should be noted that the greater part of the participants lived in the dorms on campus or apartments associated with a host university, and had little contact with a “real host life”. In the post-SA interviews, the majority admitted that they had spent most of their time in an international academic environment and, consequently, felt adapted to the “international students’ life”, rather than to the “real life” of hosts. Lola’s comment drawn from the post-SA interview is representative: “Well, I didn’t have a real Dutch life, because I didn’t work. Probably I have adapted well to the Erasmus life in Holland, not to the Dutch life.” When comparing the trajectories of those who lived on-campus and off-campus, the latter ones show more negative \isi{inflection} points. These students felt at times more distressed at such inconveniences, in their point of view, as poor public transport, unhealthy eating habits, inconvenient local timetable, and insufficient sanitary arrangements. The excerpt below illustrates Daniel’s feelings in the US:

\begin{quote}
Narrative diary (at week 2)
\smallskip\\
I knew that Americans weren’t much healthy in food terms but now I have understood it on my own flesh why it is so, and it shocks me that such a developed country has such bad habits. [...] Apart from that, distances are a problem if you don’t have a car, since public transportation is not as good as it is in Barcelona.
\end{quote}

As we have seen, personal life is tied intrinsically to the participants’ academic life and, therefore, in the following section, we present those vital experiences that have been identified by the participants as relevant during their academic stay.  


 
\subsection{Academic life}

In this section (see \tabref{tab:pogorelova:3}), we present the main themes related to \isi{inflection} points in the students’ adaptation curves for their Academic Life. \tabref{tab:pogorelova:3} below includes the number of participants for whom a life experience related to a given academic issue resulted in either a positive or negative \isi{inflection} point in their stays. It is important to notice that the students’ curves built for their Academic Life experiences showed more fluctuations than the curves for Personal Life.

\begin{table}[t]
\begin{tabularx}{\textwidth}{Qr}
\lsptoprule
 \bfseries academic issues & \bfseries \# of participants (n=12)\\
 \midrule
Academic support and enrolment process upon arrival in the host university & 5\\
\tablevspace 
Host university’s welcome events and services for integration & 8\\
\tablevspace 
Instructors and assistants assigned to resolve doubts in the academic sphere & 6\\
\tablevspace 
University facilities & 6\\
\tablevspace 
Curriculum & 6\\
\tablevspace 
High demanding requirements for assignment submissions and exams & 2\\
\tablevspace 
Distribution of workload & 2\\
\tablevspace 
Teaching methodology and working pace in class & 7\\
\tablevspace 
Expectations regarding academic experiences & 6\\
\tablevspace 
Working jointly on projects with groupmates & 4\\
\tablevspace 
Participating in classroom activities & 8\\
\tablevspace 
Relationships with lecturers & 2\\
\tablevspace 
Feedback from teachers & 6\\
\tablevspace 
Feeling improvement in the \isi{foreign language} & 8\\
\lspbottomrule
\end{tabularx} 
\caption{Main themes relating to inflection points in participants’ academic life}
\label{tab:pogorelova:3}
\end{table}

As regards academic support, the majority of participants were satisfied with the service they had received from administrative staff at their host universities. The participants assessed positively, not only welcome events and other activities organized by the mobility office for exchange students, but also the instructors who had been assigned to settle their queries in the academic sphere, as in the case of Anna, who highlighted the role of her international assistant (IA) during her stay in Boston. Her IA was an American master student who helped her resolve her doubts not only about academic aspects, but also about basic needs (e.g. where to buy winter clothes). Negative \isi{inflection} points occurred in initial stages of the SA and were mostly due to a slow process of enrolment upon arrival in the host university and bureaucratic delays in paperwork procedures.  

When describing the host educational system, the participants compared their home and host universities in terms of facilities, requirements, schedule, curriculum, and teaching approaches. In the first few weeks, the majority evaluated positively their host university’s campus facilities, such as libraries, computer labs and sporting clubs, as well as the curriculum, which was often characterized as being well organized and efficient with a wide range of courses available. However, for some, coinciding with the U-curve explanatory model, their initial excitement with the host university was followed by difficulties adjusting to new rules and requirements for assignment submissions and exams. The distribution of the workload differed from what they were used to at their home university and, as a result, final papers and projects required much more commitment. Some students ended up disliking the host educational system as a whole, which, for instance in the case of Angela, was too structured and did not allow for flexibility:

\begin{quote}
Narrative diary (at week 8)
\smallskip\\
I don’t like the way they teach in China. It is very repetitive and sometimes childish: they work with merits and if you show that you have done your homework, if you are always on time and never miss class… then you will pass the course with higher marks. It’s totally contrary to that in Europe, when you are supposed to be more independent, more curious and it is always better if you make your own questions related to your personal interests.
\end{quote}

Some had trouble adjusting to working pace in class and felt disappointed with the teaching approaches customary at the host university. For instance, Anna, whose host destination was Boston, found the teaching methodology to be of regular quality and classes as “light”. It is worth noting that participants’ initial disappointment was often caused by their very high pre-departure expectations concerning the \isi{academic experience}. In fact, six out of twelve participants reported their dissatisfaction with the \isi{academic experience} as the sojourn proceeded, and related their negative self-assessment to the unfulfillment of their pre-departure expectations, as illustrated in the following extract drawn from Anna’s diary: 

\begin{quote}
Narrative diary (at week 4)
\smallskip\\
Maybe it is because I expected so much and from there [\textit{Barcelona}] people 'venerate' the American system (without having been here I think) ... Classes are not bad but I expected much more level. Teachers are not really teachers. They comment the subject matters - not teach them.
\end{quote}

 Another issue raised by several participants was the difficulty of building closer relationships with lecturers. Sara, for instance, pointed out that her university was extremely huge, seminars were overcrowded and lecturers often changed within the same subject. She believed that such organization had affected negatively the overall atmosphere of classes and did not contribute to learning: 

\largerpage
\begin{quote}
Interview (after SA) 
\smallskip\\
It is not that I don’t like the university, but I think that the price Australian people pay for this education is really high. [...] The good thing about it is that it has many subjects to choose from, which we don’t have here. But then subjects too easy, low-level and, well, ordinary. Education at [\textit{name} \textit{of} \textit{the} \textit{home} \textit{university}] is, in some things, much better. It is a good university and it’s small, while in Sydney it was a big university. Here I have very close contacts, I know the teachers. There with the professor of philosophy yes, but with others not. They give huge classes, in one subject the teacher was changed every week and the topic was explained so badly. I don’t know, it doesn’t motivate me. 
\end{quote}

When it came to group work, the main difficulty was related to working jointly on projects and presentations with their mates. Some students found that hosts were avoiding collaboration with exchange students. Others faced a similar problem with international teammates, and reported that those were reluctant to fulfil their duties and act in concert with other group members, which forced them to take on extra work. Natalia’s comment reflects her negative feeling:

\begin{quote}
Narrative diary (at week 12)
\smallskip\\
I am so pissed off in one of my projects as the host nationals really didn’t do anything, lots of free riders. I am supposed to be the one partying around and having fun as I am the exchange, but, on the contrary, I am doing the work of others. TOO BAD!
\end{quote}

It is worth noting that the \isi{foreign language} used in the academic sphere presented a considerable obstacle for most of the students despite their high \isi{linguistic} \isi{proficiency}, which resulted in negative \isi{inflection} points in their curves. The participants articulated difficulties in contributing to classroom discussions, completing writing tasks, giving oral presentations, understanding lecturers, and reading academic literature. For instance, Ares felt \isi{initial anxiety}, because she could not contribute actively to classroom discussions and her teachers showed her their dissatisfaction:

\begin{quote}
Narrative diary (at week 4)
\smallskip\\
The worst thing I had to face is that lecturers want us to talk, ask and comment during the class. This is very new for me, because in my home university this is something completely optional and at any case mandatory, just positive. So, I had to deal with the fact that a professor was disappointed because I didn’t express my opinion during the class.
\end{quote}

 Anna and Daniel reported that their speaking skills had been their weakest point. When required to participate in classroom debates, they felt unable to develop their ideas properly due to vocabulary limitations. Natalia realized that she was not as confident as she expected when giving oral presentations. Ares faced the same problem, and she felt particularly flustered when she needed to speak in front of students with a higher level of \ili{English} or native speakers. Lola experienced difficulties understanding some of her lecturers and found completing writing tasks linguistically challenging. Similarly, Virginia found writing in French difficult and had to invest extra time correcting her French handouts. Similar to Lola and Virginia, Sara had difficulties writing essays and, as she lacked academic vocabulary, spent a lot of time translating assigned articles. The excerpt below illustrates Sara’s feelings of anxiety:

\begin{quote}
Narrative diary (at week 6)
\smallskip\\
Now I understand almost everything the professors say but I am a little bit “crushed” because for the next week I have to write an essay in philosophy of about four pages and my \ili{English} is still regular for writing about philosophy. I feel very childish when writing, I know that I make mistakes and I get frustrated!\footnote{Original quote in \ili{Spanish}: “Ahora entiendo casi todo lo que dicen los profes pero estoy un poco “rallada” \textit{[sic]} porque para la semana que viene tengo que hacer un ensayo de filosofía de unas cuatro páginas y mi inglés ya es regular como para encima escribir filosofía. Me siento muy infantil cuando escribo, sé que hago muchas faltas y me frustro!”}
\end{quote}

 As can be judged from their narratives, most students felt a great improvement in their language skills as the sojourn proceeded. They often appreciated extensive feedback from their teachers on their assignments and considered it as being beneficial for their learning, especially in terms of their academic language use. Following classes, submitting written tasks, giving presentations, reading assigned articles and, most importantly, passing exams in a \isi{foreign language} not only contributed to the development of the corresponding skills, but also to the students’ \isi{linguistic} confidence and pride in accomplishments, similarly to \citegen{Beaven2012} \ili{Italian} participants. Therefore, the positive \isi{inflection} points in the adaptation curves for the academic life resulted from good grades, encouraging comments from teachers on their exams, and satisfaction with achieved outcomes.

All in all, these results point to the diversity and complexity of both life experiences and individual, social, \isi{linguistic} and academic abilities that may affect adaptation during an academic stay.   


\section{Discussion}
\largerpage
This chapter has attempted to shed light onto those vital experiences that may influence the adaptation process of bilingual undergraduate students during a SA.

As regards personal experiences, friends from home and family provided emotional support and their sporadic visits positively influenced the participants’ personal lives, confirming \citegen{Beaven2012} results. The relationships with co-nation\-als who were in the same location also had a positive influence, especially during stressful moments. \citet{Pitts2009} also argues that co-national support plays an important role for adjustment in the context of short-term mobility, a phenomenon further confirmed by Coleman (2013, 2015). Since in this study these experiences always coincided with significant, positive changes in their self-assessment, we would rather go beyond the concept of “emotional support” and argue that, in terms of language and daily cultural habits, co-nationals provide a “zone of familiar comfort” which acts as a fulcrum for the students to confront the new milieu with self-assurance and strength.

Meeting people from the host country was one of the main motives for going abroad and it turned out a key issue in the adaptation process. In fact, most participants felt more satisfied with the relationships they could build with other exchange students and considered them as beneficial for their \isi{intercultural} awareness. In contrast, the greater part of the students had trouble in establishing friendly relationships with hosts, with \isi{linguistic} difficulties being a major hurdle, at least initially. Even Cristina, who did medical practices in \ili{Spanish}-speaking Argentina and needed to speak neither \ili{English} nor other foreign languages, reported difficulties understanding the local dialect at the beginning of her SA. In fact, such correlation between \isi{linguistic} \isi{fluency} and increased \isi{intercultural} interactions has already been explored and reflected in the culture-learning model \citep{Ward2004}. Despite the fact that SA programmes are developed to bring students in closer contact with a host society, the Erasmus context itself may complicate contact with host nationals both at the university and in the residence halls. Reluctance on the part of host nationals was mentioned by our participants as an impediment to building friendships with them. Similarly, \citet[190]{Ward2004} claims that such \isi{intercultural} friendships tend to be infrequent due to various factors, “including the willingness of host nationals to interact across cultural boundaries and their perceptions of newcomers.” In our view, if students on an SA are not prepared before their departure to expect such attitudes, their unfruitful efforts towards building a network of host friends may lead precisely to such negative experiences as feeling homesick —experiences which that network is supposed to prevent \citep{HendricksonEtAl2011}— or little empathic towards the host culture, in the terms of \citegen{Bennett1993} Developmental Model of Intercultural Sensitivity (DMIS). For the purpose of supporting and helping Erasmus students benefit from their \isi{intercultural} encounters, the IEREST\footnote{Intercultural education resources for Erasmus students and their teachers. Koper: Annales University Press. Retrieved from http://ierest-project.eu/humbox} (2015) offers a wide range of teaching resources and practical guidelines that span the three stages (before, during and after) of the SA experience.  

Given the positive effect on adaptation of \isi{intercultural} friendships, we pose that the definition of the term “\isi{intercultural} adaptation” in the SA context should be extended by moving from the restricted meaning of  “adaptation to a host culture” (as if the sojourner was just encountering one new culture in the SA context) to the wider meaning of  “adaptation to foreign cultures,” a definition that would include, not only the host culture, but also those brought to the SA context by other international students. By reconsidering the concept of adaptation, we could claim that SA nowadays could be preparing students for life afterwards in increasingly multicultural societies, whether in their own home country or in another country of residence. For students who have shared primary and secondary classrooms with people from culturally different backgrounds, the context of a university SA would reinforce pre-existing, if any, \isi{intercultural} abilities and act as a springboard for adult life beyond the academic sphere. 

Accommodation during the sojourn is one more important characteristic that needs to be considered (\citealt{VandeBergEtAl2009}). In their daily life, those participants who lived on campus did not find their new environment to be much different from their hometown and seemed to experience no trouble adjusting to it. However, as was mentioned earlier, they had little contact with other social domains/environments in the host culture. Residing either in apartments or dorms, which were normally situated on the outskirts of a city and fully equipped, often made it unnecessary for them to leave this comfortable academic/international area. Thus, it would be appropriate to say that they adjusted to the international student lifestyle and their successful adaptation processes can be discussed only in the particular on-campus context. 

The relevance of context in the adaptation process resulted, in fact, in two quite different SA experiences: what we may call “predominantly on-campus SA” (i.e. when residing in residence halls and apartments with other university, mostly international, students) and a “predominantly off-campus SA” (i.e. when residing outside the university). For our participants, the off-campus residence revealed daily difficulties, not felt by those living on campus, when adjusting to certain inconveniences, such as poor public transport, unhealthy food, bad sanitary conditions, and local timetable. Such difficulties translated into negative \isi{inflection} points in their adaptation cycle. In her dissertation, Beaven suggests that students may face difficulties when moving from on-campus to off-campus facilities and their “adaptation cycle may start all over again as the changed environment makes new demands on the sojourner” (2012: 286). Beyond the mere location of the residence and the existence or not of the network of local friends, at least two other SA conditions would be at play to make the on-campus SA similar to a more “real” —in the participants’ terms— off-campus one: 1) work experience off-campus and 2) extensive travelling, which allowed our participants to see the host destination under a different light. All in all, such difficulties highlight the necessity of pre-departure preparation sessions in which students should be briefed on possible challenges linked to different SA conditions.

If we now turn to academic life, the results show that academic experiences triggered \isi{inflection} points in the participants’ adaptation curves more often than personal ones. Linguistic difficulties often translated into negative self-assess\-ment, especially during the first half of the sojourn. However, most participants eventually perceived they had improved their \isi{foreign language} skills, especially their listening, reading and writing. This lends support to previous research studies that report the beneficial effect of the SA context on the development of such skills (\citealt{Pérez-Vidal2014}). In fact, both Bennett’s and Ward’s models point at the \isi{acquisition} of necessary communication skills for effective \isi{intercultural} communication. As a result of the SA, some participants felt that they had become even more willing to learn foreign languages, which is in line with research that reports a positive effect of SA on motivation towards language learning \citep{Trenchs-PareraJuan-Garau2014}. The connection between reduced anxiety and increased \isi{linguistic} confidence found by Trenchs-Parera and Juan-Garau in their quantitative study was confirmed in ours, as the \isi{initial anxiety} that some participants experienced decreased. 

Nevertheless, when commenting on their academic outcomes, only half of the participants felt that they had benefited from the chosen courses, contradicting previous studies that report the overall academic improvement of their participants after the stay \citep{Teichler2004}. The other half expressed disappointment with the academic side of their SA experience. As main reasons, they mentioned (a) strictness and inflexibility of the host educational system, (b) repetitive and easy classes that lacked deep analysis, and (c) unexceptional teaching methodology that did not correlate with the host university’s prestige and tuition fees. This frustration points at a gap between reality and very high pre-departure expectations, as also documented by \citet{Ward2004}. Our participants’ expectations were mostly based on the assumption that foreign universities were much better, but the experience taught them to value their home university higher. Therefore, confronting a foreign academic environment had a positive “boomerang effect” and triggered a positive, conscious re-assessment of their own academic culture.

Relationships with professors abroad also hindered adaptation to the new environment for some students who often blamed it on the number of students in seminars and the changing of teachers within subjects, as several of Beaven’s participants (2012) had reported. A possible explanation to such a reaction may lie in the fact that our participants’ home university is far smaller than some of the host universities and registers fewer people in classes. Besides, the degree of formality between teachers and students in \ili{Catalan} universities is less strict than in other countries, and students are accustomed to behaving more informally with their professors. It should be noted that, even though we are reporting this phenomenon in the context of SA, it could also take place locally, for instance when changing universities. Thus, an SA during undergraduate studies may be serving an unexpected academic goal: namely, preparing students for better adjustment when facing academic differences in local, supposedly familiar contexts.

All in all, these experiences point to the complexity of the adaptation process, to interactions among a large number of aspects —such as pre-departure expectations, co-national support, or individual academic and \isi{linguistic} abilities— and to the beneficial effects of pre-departure preparation. 


\section{Conclusion}

Although the current study is based on a small sample of participants, the findings are highly informative for study abroad administrators and instructors. In such societies as Catalonia, it is believed that bilingual and \isi{multilingual} people adapt easily to new cultural environments because of their \isi{linguistic} flexibility and, often, because internationalized secondary schools provide a kind of preparatory context for encounters with other cultures. However, the \isi{inflection} points in our participants’ life experiences during SA and their effect on their \isi{intercultural} adaptation do not reveal major differences between our bilingual \ili{Catalan} participants and monolingual ones in previous —both quantitative and qualitative— studies (\citealt{Williams2005,VandeBergEtAl2009,Beaven2012}). As a token, the \isi{foreign language} used in the academic sphere presented a considerable obstacle for most of our participants, even if their \isi{linguistic} abilities varied between B2 and C1 levels. What our study has revealed is more nuances in already identified phenomena, such as the role of co-nationals as an emotional fulcrum, simultaneous adaptation to a multiplicity of cultures, the role of different lodgings as triggers of varied SA experiences, the unexpected difficult adaptation to the variety of one’s native language spoken in the host culture, and difficulties to adapt to a different academic culture that may also occur in the home country and not be exclusive of an SA.

\section*{Acknowledgements}

We gratefully acknowledge our twelve participants who provided us with rich data for this study as well as the Deans and Mobility Coordinators in the Faculties of Health and Life Sciences, Political and Social Sciences, Humanities and Communication of the \ili{Catalan} university for facilitating contacts with outgoing students. We would also like to thank the technical staff for their help in creating Moodle classrooms and Roberto Molowny-Horas for his help with statistical analysis. The present work has benefitted from funding from UPF’s research group GREILI and from the research project {FFI2014-52663-P}{ }{(Spain’s} Ministry of Economy and Competivity){. }Finally, special thanks go to this volume’s editors and the anonymous reviewer for their feedback on earlier versions of this chapter. 
 
\sloppy
\printbibliography[heading=subbibliography,notkeyword=this] 
\end{document}