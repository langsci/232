\documentclass[output=paper]{langsci/langscibook} 
\author{Ariadna Sánchez-Hernández\affiliation{Jaume I University}}
\title{Acculturation and pragmatic learning: International students in the United States} 
\shorttitlerunninghead{Acculturation and pragmatic learning: International students in the US} 
 
\abstract{The present study explores the relationship between acculturation and the development of second language (L2) pragmatic competence during a semester-long study abroad (SA) program in the United States (US). Drawing on \citegen{Schumann1978} Acculturation theory of L2 acquisition, it was hypothesized that the degree to which SA participants acculturate socially and psychologically to the target language community would be related to the extent to which they acquire L2 pragmatic competence. Twelve international students of three different nationalities – Brazilian, Turkish, and Spanish – in their first semester of study at an American university completed a pre-test and a post-test version of a discourse completion task that measured their ability to produce speech acts and a sociocultural adaptation scale \citep{WardKennedy1999} that measured their acculturation. Additionally, they participated in semi-structured interviews at the beginning and at the end of the stay that provided insights into their SA adaptation experiences. An exploration of individual trajectories indicated that gains in pragmatic competence were promoted by acculturation development. On the one hand, pragmatic gains were related to social variables that included the integration strategy adopted and academic pressure. On the other hand, they were associated with affective factors such as social support from home-country peers. The reported findings bring new insights to the field of L2 pragmatics by examining the effects of acculturation. Ultimately, the results emphasize the importance of enhancing L2 learners’ social and affective adaptation during SA programs, so as to maximize their acculturation experiences and their subsequent L2 pragmatic learning.}
\maketitle


\begin{document}
\section{Introduction}
A main consequence of globalization is the increase of study abroad (SA) programs all over the world, which have even become mandatory for many university students. However, the traditional view that SA programs are the optimal context for learning\footnote{Acknowledging the difference between the terms \textit{acquisition} and \textit{learning} pointed out by \citet{Krashen1985} - i.e., natural \isi{acquisition} vs. \isi{acquisition} that involves \isi{formal instruction} respectively – the present study follows the mainstream use of both terms as synonyms to refer to language development.}  a second language (\isi{L2}) is being challenged by studies reporting cases of unsuccessful adaptation experiences by international students (for a review, see \citealt{MitchellEtAl2015,MitchellEtAl2017}) and limited \isi{acquisition} of some \isi{pragmatic} features (for a review, see \citealt{TaguchiRoever2017}). This is not surprising if one considers that SA participants not only have to focus on improving their \isi{L2} \isi{proficiency}, but also have to face the multiple challenges involved in the process of adapting to a new and unknown setting, while being expected to interact with people of diverse \isi{sociocultural} backgrounds. Drawing on this idea, recent research (e.g. \citealt{TaguchiEtAl2016}; \citealt{Sykes2017}; \citealt{Taguchi2017}; \citealt{TaguchiRoever2017}) points out that a problem in understanding SA outcomes is that there is a scarcity of empirical support for the relationship between \isi{intercultural} and \isi{pragmatic} competences, areas that have traditionally belonged to different domains (Psychology and \isi{Linguistics}, respectively). 

 
To address this problem, the present study explores the extent to which SA participants’ \isi{acculturation} experiences are related to the development of their L2 \isi{pragmatic} competence in the SA context. Drawing on Schumann’s (\citeyear {Schumann1978}; \citeyear{Schumann1986}) Acculturation theory of \isi{L2} \isi{acquisition}, the study is based on the premise that the degree to which an individual acculturates to the \isi{target language} society will determine his/her \isi{acquisition} of the \isi{L2}. According to \citet{Schumann1978}, in the process of adapting to a new environment, different social variables (e.g. integration strategies, attitude towards the host culture) and affective factors (e.g. \isi{culture shock}, motivation) are at play. While the Acculturation model has commonly been used in the general field of \isi{L2} \isi{acquisition} (e.g. \citealt{Hansen1995}; \citealt{Lybeck2002}), its application in the field of \isi{L2} \isi{pragmatics} still represents a research desideratum. There is no conclusive evidence as to whether \isi{acquisition} of L2 \isi{pragmatic} ability during a stay abroad is related to students’ \isi{acculturation} experiences.
 
\section{Literature review}
\largerpage
\subsection{Study abroad programs as a context for learning pragmatics}
Study abroad programs – that is, temporary educational sojourns in which a \isi{target language} is used by the members of the community \citep{Taguchi2015contextually} – have typically been referred to as the optimal context for the \isi{acquisition} of \isi{pragmatic} competence. Mastering \isi{pragmatic} competence in a \isi{L2} involves learning how to the use language appropriate in the context, the situation and with the interlocutors – in other words, knowing “\textit{when} and \textit{where} to say something, \textit{what} to say, [and] to \textit{whom} to say it in a given social and \isi{linguistic} context” \citep[314]{García1989}. Pragmatic ability mainly involves the ability to perform speech acts, such as suggestions, requests, refusals, apologies, and compliments, among others. In addition, it concerns the mastery of \isi{pragmatic} further features like \isi{implied meaning}, \isi{pragmatic} routines (i.e., \isi{formulaic} language recurrently used by native speakers (\isi{NSs}) in given situations), and managing interaction (i.e., turn-taking or conversation openings).  While studying abroad, learners are likely to acquire these features as they have the potential to have rich exposure to the \isi{L2} outside of class and plenty of opportunities to use the language in diverse social situations, with different interlocutors and for real-life purposes. Moreover, they continuously witness interactions among users of the \isi{L2} that provide them with valuable and authentic input.  \citet[4]{Taguchi2015contextually} summarizes the main elements that make the SA context potentially optimal for L2 \isi{pragmatic} learning as given in (1).

\ea%0
  \ea Opportunities to observe local norms of interaction; 
  \ex contextualized \isi{pragmatic} practice and immediate feedback on that practice;
  \ex real-life consequences of \isi{pragmatic} behavior; and 
  \ex exposure to variation in styles and communicative situations.
  \z
\z
    
There is a burgeoning of studies in the field of \isi{interlanguage} \isi{pragmatics} (ILP) that have pointed out the advantage of the SA context for the development of different \isi{pragmatic} features. These have typically involved cross-sectional investigations, that is, comparing \isi{pragmatic} ability among groups of \isi{L2} learners or comparing  \isi{NSs} and non-native speakers ({NNSs}), or, to a lesser extent, longitudinal studies that examine \isi{pragmatic} development over time (see \citealt{Alcón-Soler2014} for a review of cross-sectional and longitudinal ILP findings in the SA context). All in all, these studies have reported that during SA, learners improve their \isi{pragmatic} awareness, their production of speech acts, their use of \isi{pragmatic} routines and their comprehension of \isi{implied meaning}. 

Nevertheless, longitudinal ILP studies have shown that the advantage of the SA context for \isi{pragmatic} development is not straightforward. The process of acquiring \isi{L2} \isi{pragmatic} competence is variable and non-linear, as it depends on (1) the \isi{pragmatic} feature under study and (2) different factors associated with the SA setting. Regarding the role of the pragmatic feature under study, SA seems to be beneficial for the \isi{acquisition} of \isi{pragmatic} routines, but there are mixed findings on its benefits for the ability to comprehend \isi{implied meaning} and to produce certain speech acts (\citealt{TaguchiRoever2017}). Indeed, not all speech acts present the same degree of difficulty. For instance, greetings, leave-takings, and offers are acquired more quickly, and students thus learn them at earlier stages of \isi{immersion}, while appropriate use of requests, refusals, and invitations is achieved at a slower pace and is thus more common in longer SA sojourns (\citealt{Barron2003}; \citealt{Félix-Brasdefer2004}; \citealt{Hassall2006}).

As for the different factors associated with the SA setting ILP scholars have commonly classified predictors of \isi{pragmatic} learning during SA into two main categories: external factors related to the context, and internal ones related to learners’ individual differences. The main external factors investigated in ILP research are length of stay and intensity of interaction with users of the \isi{L2}, with studies reporting that amount of interaction is a better predictor of \isi{pragmatic} development than length of stay (e.g. \citealt{Bardovi-HarligBastos2011}; \citealt{Bella2011}). That is, spending more time in the \isi{target language} setting is not enough on its own to fully develop \isi{pragmatic} competence, as \isi{L2} learners need to be willing to take advantage of the opportunities for interaction offered by the context. Nevertheless, a focus on the role of external factors does not seem to be enough to explain \isi{L2} \isi{pragmatic} \isi{acquisition}, since internal factors often interfere with the effect of contextual variables. Evidence of this fact is provided, for example, by \citet{EslamiAhn2014}, who explored how L2 \isi{pragmatic} development (measured in terms of the ability to respond to compliments) by \ili{Korean} students in the US was influenced by two external factors, namely length of stay and intensity of interaction and one internal variable, namely motivation, reporting that only motivation had a positive impact on \isi{pragmatic} development. All in all, \isi{proficiency} has been the most investigated internal predictor of \isi{pragmatic} \isi{acquisition}, with most research findings indicating that having a certain \isi{proficiency} level enhances the \isi{acquisition} of most \isi{pragmatic} features, although lower-level students at times outperform higher-level ones depending on the \isi{pragmatic} feature and on the context (for a review, see \citealt{Xiao2015}).

\largerpage
In sum, although most ILP investigations have revealed positive gains in L2 \isi{pragmatic} ability during SA, they have also reported that such L2 \isi{pragmatic} development is variable and non-linear, as it is influenced by different factors. Drawing on this idea, some scholars (e.g. \citealt{Taguchi2015contextually}) have expressed the need for ILP studies to focus on the processes – rather than merely on the outcomes – of SA. This implies a call for more longitudinal research on the factors that influence the development of \isi{pragmatic} ability over time. More particularly, in a recent monograph, \citet{TaguchiRoever2017} call for ILP studies to investigate new variables that have gained importance in the current era of globalization, such as \isi{intercultural} competence, an umbrella term that encompasses the concept of \isi{acculturation}.

\subsection{Acculturation and pragmatic learning}\label{sec:sanchez:2.2}

The present study aims to bridge the gap between internal and external factors that affect \isi{L2} \isi{pragmatic} learning by focusing on the variable of \isi{acculturation}. {Acculturation} is a multifold phenomenon that is defined as “the process of cultural change that occurs when individuals from different cultural backgrounds come into prolonged, continuous, first-hand contact with each other” (\citealt{RedfieldEtAl1936}: 146). It has been operationalized in terms of three main constructs: \isi{acculturation} conditions (antecedent factors such as the characteristics of the sojourning and host cultures, of the sojourning group, and of the individuals), \isi{acculturation} orientations (strategies of integration in the host society, such as assimilation, marginalization, or separation), and \isi{acculturation} outcomes, which include \isi{sociocultural} adaptation (implying behavioral aspects, skills, attitudes, and cultural knowledge) and \isi{psychological adaptation} (sojourners’ well-being and satisfaction) (see \citealt{Arends-TóthVanDeVijver2006}). 

Different models have been proposed in an attempt to explore the influence of \isi{acculturation} on the \isi{acquisition} of an \isi{L2}. Three major frameworks include the Inter-group model by \citet{BeebeGiles1984}, the Socio-Educational model by Gardner \citep{GardnerEtAl1983}, and the Acculturation model by Schumann (\citeyear{Schumann1978}; \citeyear{Schumann1986}) (see \citealt{Ellis1994}: Chapter 3). The present study takes Schumann’s model as a reference to understand \isi{acculturation}, as it accounts for \isi{acculturation} conditions, orientations and outcomes. Moreover, it is the only model that has generated empirical evidence concerning the relationship between \isi{acculturation} and \isi{pragmatic} learning \citep{Schmidt1983,DörnyeiEtAl2004,SchmittEtAl2004}.

In the first seminal book that provides a comprehensive review of \isi{L2} \isi{pragmatic} development, \citet{KasperRose2002} present \citegen{Schumann1978} Acculturation model as the first theoretical framework that explains \isi{pragmatic} development. According to Schumann, the degree to which an \isi{L2} learner acculturates to the new \isi{sociocultural} community will influence the extent to which he/she learns the \isi{target language}, \isi{acculturation} being the first (but not the only one) in a list of factors that determines \isi{L2} \isi{acquisition}. A key point of Schumann’s theory is that \isi{acculturation} is determined by the proximity of the sojourner to the \isi{target language} group in terms of \isi{sociocultural} and \isi{psychological adaptation}. Sociocultural adaptation refers to the degree to which a language learner achieves contact with the \isi{L2} group and becomes part of it; it thus depends on the individual’s skills with respect to integration and management of everyday situations. Psychological adaptation involves the degree to which a student is comfortable with the learning and adaptation processes and therefore implies emotional well-being and personal satisfaction. To determine the extent of \isi{acculturation} with respect to these two aspects, \citet{Schumann1986} distinguished two sets of factors. Firstly, seven social factors, provided in (2), shape \isi{sociocultural} adaptation: 

\ea%1
\ea Social dominance of the \isi{target language} group, in terms of political, cultural, technical, and economic status, as perceived by the sojourning group.
\ex Integration strategy: assimilation, preservation or adaptation of \isi{sociocultural} values.
\ex Enclosure: the degree to which the two cultural groups share the same social facilities.
\ex Cohesiveness and size of the sojourning group.
\ex Cultural congruence between the two groups regarding religion, general social practice, and other beliefs.
\ex Attitude towards the host culture.
\ex Intended length of stay in the \isi{target language} context.
\z
\z

Secondly, the four affective factors in (3) determine \isi{psychological adaptation}: 

\ea%2
\ea  Language shock: fear of appearing ridiculous when speaking the \isi{L2}.
\ex Culture shock: feelings of rejection, anxiety, and disorientation by the sojourners while living amongst members of the target community.
\ex Motivation: according to Schumann, an \isi{integrative motivation} is more likely to assist in \isi{SLA} than an instrumental one. 
\ex Ego permeability: the extent to which identity is flexible and can adapt.
\z
\z

A few studies have drawn on \citegen{Schumann1978} assertion that the degree to which individuals \isi{acculturate} will determine the extent to which they learn the \isi{L2}. Most of them suggest that \isi{L2} \isi{acquisition}, especially in terms of oral \isi{proficiency}, is benefited by the learners’ process of \isi{acculturation} (\citealt{Hansen1995}; \citealt{Lybeck2002}; \citealt{JiangEtAl2009}). In the field of \isi{pragmatics}, \citegen{Schmidt1983} and \cite{DörnyeiEtAl2004}, are, to the best of my knowledge, that have applied Schumann’s model to explain \isi{L2} \isi{pragmatic} development.

\citet{Schmidt1983} conducted a case study of Wes, a 33-year-old \ili{Japanese} male who immigrated to the US (Hawaii) without having previous \isi{formal instruction} in \ili{English}. Wes’s development with respect to \isi{acculturation} and \isi{L2} \isi{acquisition} was tracked over 3 years. Having the optimal \isi{sociocultural} and psychological orientations, he increased his \isi{pragmatic} ability but did not improve his \isi{grammatical} competence. To assess \isi{pragmatic} competence, Schmidt focused on directives, which include speech acts used to incite action on the part of the interlocutor, such as orders, requests, and suggestions. At earlier stages of \isi{pragmatic} development, Wes’s use of directives was characterized by a reliance on a small number of speech formulas that he only used in specific situations (for example, \textit{shall we go?}), and by transfer from \ili{Japanese} sociopragmatic norms. Over time, he improved the \isi{appropriateness} of meanings, reduced \isi{pragmatic} transfer, became aware of differences between languages, and developed significant control of \isi{speech act} strategies and formulas used in social interactions. \cite{Schmidt1983} thus confirmed that \isi{acculturation} led to increased \isi{L2} \isi{pragmatic} competence.

\citet{SchmittEtAl2004} analyzed quantitatively how \isi{acculturation} affected the acquisition of\isi{formulaic} language learning. They quickly realized that \isi{acculturation} was a complex phenomenon that demanded a qualitative in-depth analysis, which led them to conduct semi-structured interviews with a subset of seven of the participants. This second investigation, conducted by \citet{DörnyeiEtAl2004}, was a case study of seven international students having spent seven months in a British university, in which the authors explored the participants’ \isi{acculturation} development in terms of \isi{sociocultural} adaptation, measured through the social factors outlined in \citegen{Schumann1978} model\footnote{\citet[88]{DörnyeiEtAl2004} take \citegen{Schumann1978} Acculturation theory as a base, but they focus on the social aspects of the process. They define \isi{acculturation} as “the extent to which learners succeeded in settling in and engaging with the host community, thereby taking advantage of the social contact opportunities available”.}. Four of the participants showed positive gains in their ability to use \isi{formulaic} language, while three of them did not experience such gains. Research findings indicated a strong relationship between \isi{sociocultural} adaptation and \isi{pragmatic} learning. In particular, \isi{acquisition} of \isi{formulaic} language was mainly influenced by the variables of enclosure and the \isi{integration strategy} adopted, as evidenced in the participants’ development of social networks. Indeed, most of the participants found it extremely hard to have meaningful contact with the \isi{L2} speakers outside of class. Successful learning of \isi{formulaic} language depended on whether they could “beat the odds” and come out of the “international ghetto” \citep[105]{DörnyeiEtAl2004}. This was evident in two of the four participants, who scored higher in a \isi{formulaic} language test. The other two successful students had extraordinary motivation and language aptitude. These two aspects therefore also played a key role in \isi{pragmatic} learning. 

\largerpage
\citegen{Schumann1978} Acculturation theory, however, has received little further empirical support, and it has faced some criticism (\citealt{Ellis1994}; \citealt{Zaker2016}). The main critique has been that it is difficult to assess some of the variables proposed by Schumann. Moreover, the framework disregards additional factors that may be better predictors of \isi{L2} learning, such as individual differences (e.g. cognitive abilities, learning style) and instruction (\citealt{Mondy2007}; cited in \citealt{Zaker2016}). According to  (\citeyear{Mondy2007}) learners may \isi{acculturate} successfully despite not having favorable conditions in the social and affective variables proposed by Schumann. 

Other studies have explored the role of \isi{acculturation} factors on \isi{pragmatic} development without drawing on \citegen{Schumann1978} model. Overall, they have reported that \isi{pragmatic} competence is determined by specific aspects such as identity \citep{Siegal1995}, motivation (\citealt{EslamiAhn2014}), and cultural similarity (\citealt{Bardovi-HarligEtAl2008}). Additionally, a recent line of studies has addressed the role of \isi{intercultural} competence on the development of \isi{pragmatic} ability (\citealt{Taguchi2015crosscultural}; \citealt{TaguchiEtAl2016}). Nevertheless, the question still remains as to whether the development of \isi{pragmatic} competence is determined by students’ \isi{sociocultural} and \isi{psychological adaptation} during SA. The current study directly addresses this question and in doing so fills in the existing gap between studies on \isi{acculturation} and on \isi{L2} \isi{pragmatic} \isi{acquisition}.

\section{Research questions}
\largerpage
This investigation sheds new light on the development of \isi{L2} \isi{pragmatic} competence in the SA context by exploring the influence of \isi{acculturation} on the \isi{development of speech act} performance by students in their first semester of participation in a SA program in the US. Two research questions guide the study:

\begin{enumerate}
\item[\textbf{RQ1.}] Does a semester of study abroad afford gains in L2\isi{pragmatic} competence in terms of \isi{speech act production}?
\item[\textbf{RQ2.}] To what extent, if any, are gains in L2 \isi{pragmatic} competence related to students’ \isi{acculturation} development, measured in terms of \isi{sociocultural} and \isi{psychological adaptation}?
\end{enumerate}

\section{Method}

\subsection{Research design}

To address these research questions, a mixed-method case study approach was employed. This methodology differs from purely qualitative case-study ethnography as it integrates a quantitative research component, which in this case provided an objective assessment of \isi{pragmatic} competence and of \isi{sociocultural} adaptation. Additionally, both \isi{sociocultural} and \isi{psychological adaptation} were measured qualitatively through semi-structured interviews. Moreover, the current investigation was longitudinal and involved two data-collection points: at the beginning and at the end of a semester.


\subsection{Participants}


Twelve international students at a public university in the US Midwest participated in the study. The sample was drawn from a larger-scale study that involved 122 international students. The group of 12 was selected as they had volunteered to take part in interviews. \tabref{tab:sanchez:1} summarizes the demographic information about the 12 informants.

\begin{table}
\caption{Demographic information about case-study informants}
\small
\label{tab:sanchez:1}
\begin{tabularx}{\textwidth}{lllllQ}
\lsptoprule
\bfseries Pseudonym & \bfseries Age & \bfseries Gender & \bfseries Nationality & \bfseries Proficiency & \bfseries  Living  situation\\
\midrule 
{David}   &   {23} & {M} & {Brazilian}  &    {Beginner}       &    \mbox{With \isi{NNSs} (Brazilian)}                       \\
{Emma}    &   {26} & {F} & {Spanish}    &    {Advanced}       &    {With \isi{NNSs} (\ili{Spanish})}                         \\        
{Sean}    &   {25} & {M} & {Turkish}    &    {Advanced}       &    \mbox{Change: \isi{NSs} to \isi{NNSs}} 
\mbox{(diverse nationalities)}   \\
{Lisa}    &   {24} & {F} & {Spanish}    &    {Intermediate}   &    {With \isi{NNSs} (\ili{Spanish})}                         \\
{Jeff}    &   {20} & {M} & {Brazilian}  &    {Advanced}       &    \mbox{With \isi{NNSs} (Brazilian)}                       \\
{William} &   {22} & {M} & {Brazilian}  &    {Advanced}       &    \mbox{With \isi{NNSs} (Brazilian)}                       \\
{Steven}  &   {26} & {M} & {Turkish}    &    {Advanced}       &    {With \isi{NNSs} (diverse nationalities)}             \\
{Jason}   &   {26} & {M} & {Brazilian}  &    {Intermediate}   &    \mbox{With \isi{NNSs} (Brazilian)}                       \\
{Ethan}   &   {29} & {M} & {Spanish}    &    {Advanced}       &    {With NSs}                                    \\
{Michelle}&   {29} & {F} & {Turkish}    &    {Intermediate}   &    {With NSs}                                    \\
Mark      &   27   & M   & Turkish      &    Intermediate     &    With \isi{NSs}\\              
\lspbottomrule
\end{tabularx}
\end{table}




The group consisted of three females and nine males, their mean age was 24.4 (ranging from 20 to 29), and they were of three different nationalities: Brazilian (\textit{n} = 5), Turkish (\textit{n} = 4), and \ili{Spanish} (\textit{n} = 3). All of them were in their first semester of study in the US, and their living arrangements varied: eight of them were living with \isi{NNSs}, three of them were living with \isi{NSs}, and one student changed from living with \isi{NSs} to living with \isi{NNSs}. The amount of \ili{English} instruction they received during the semester depended on their English \isi{proficiency} level, which was measured by scores on the Test of \ili{English} as Foreign Language (TOEFL). Beginner students (\textit{n} = 1) enrolled in full-time \ili{English} classes, intermediate students (\textit{n} = 4) took part-time classes (and therefore combined them with content classes), and advanced (\textit{n} = 7) learners took occasional and specialized \ili{English} courses in addition to content classes.


\subsection{Instruments} 
Three main instruments were used in the study: a written discourse completion task (DCT) that measured students’ \isi{pragmatic} knowledge quantitatively, a modified Sociocultural Adaptation Scale (\isi{SCAS}; \citealt{WardKennedy1999}) that assessed their \isi{sociocultural} adaptation, and semi-structured interviews that revealed qualitative information about students’ \isi{acculturation} in the US in terms of \isi{sociocultural} and \isi{psychological adaptation}. Additionally, a background questionnaire was administered to collect demographic information and to control for variables such as age, \isi{proficiency}, previous experience abroad, and nationality. 

The written DCT was developed to elicit participants’ production of speech acts in high-imposition and low-imposition situations. The choice of the instrument was based on the suitability of DCTs for the elicitation of information about \isi{speech act production}, as they allow the researcher to control the given conditions and to obtain simulated oral data \citep{Félix-BrasdeferHasler-Baker2017}. The selected speech acts were requests and refusals, chosen because of the importance of their appropriate use by \isi{L2} learners for \isi{successful communication} with \isi{NSs}. Requests and refusals are considered “face-threatening acts” (\citealt{BrownLevinson1987}); inappropriate production of these could therefore lead to unintended offense by the interlocutor. \tabref{tab:sanchez:2} displays the classification of the \isi{speech act} situations included in the DCT. 


\begin{table}
\caption{Description of the situations included in the DCT}
\label{tab:sanchez:2}

\fittable{
\begin{tabular}{ll}
\lsptoprule
\multicolumn{2}{l}{\bfseries High-imposition situations}\\
\midrule
{1. Request}  &{Asking a professor for an extension of the deadline for an assignment} \\
{2. Request}  &{Asking a professor to have the test on a different day}                 \\
{3. Refusal}  &{Refusing to take summer classes}                                        \\
4. Refusal    &Refusing to help a lecturer carry some books to his/her office        \\

\tablevspace
\multicolumn{2}{l}{\bfseries Low-imposition situations} \\
\midrule 
{5. Request}& {Asking a friend for a pen}                           \\
{6. Request}& {Asking a friend for a ride to the supermarket}   \\
{7. Refusal}& {Refusing an invitation to a party}                   \\
8. Refusal &  Refusing to lend your notes to a classmate         \\
\lspbottomrule
\end{tabular}
}
\end{table}

To determine the high- and low-imposition categories, \citegen{BrownLevinson1987} framework was employed by considering the social distance between the speaker and hearer, the social power of the interlocutors, and the degree of imposition. High-imposition situations included formal interactions between a student (the speaker) and a professor (the hearer), where the social distance is large, and the social power of the hearer is higher. Low-imposition scenarios involved informal interactions between two students, where the social distance is small, and the social power of the interlocutors is equal. The selection of the situations was made on the basis of previous studies that have used DCTs to explore the production of requests and refusals (\citealt{Alcón-Soler2008}; \citealt{Taguchi2006,Taguchi2011,Taguchi2013,Martínez-FlorUsó-Juan2011}). The DCT was validated through a pilot study conducted in the previous academic semester with eight \isi{NSs} and with 21 international students enrolled at the same university. This preliminary study aimed to check whether the situations were understood correctly and whether they elicited the corresponding \isi{speech act}. 

\largerpage
Regarding the assessment of \isi{acculturation}, a modified version of the Sociocultural Adaptation Scale\footnote{The \isi{SCAS} has been widely used in empirical studies given its strong psychometric properties \citep{CelenkVanDeVijver2011}. In this study, the calculation of Cronbach alpha coefficient in the entire sample of 122 participants revealed a strong internal consistency of the scale (α = 0.937).} (\isi{SCAS}; \citealt{WardKennedy1999}) was used to measure participants’ \isi{sociocultural} adaptation in the US. Drawing from \citet{Berry2003}, I focused the quantitative analysis on \isi{sociocultural} – rather than on psychological – adaptation, considering that affective factors are to some degree responsible for students’ \isi{sociocultural} adaptation. The \isi{SCAS} is a five-point Likert scale in which students are asked to rate from 1 (= very difficult) to 5 (= no difficulty) their level of adaptation to 29 items. In the original instrument, high scores had been associated with higher levels of difficulty (that is, less degree of \isi{acculturation}). In this study, items were reversed from the original scale so that higher scores corresponded with a positive adaptation. These items included 21 behavioral situations such as \textit{finding food you enjoy} and \textit{making friends}, and seven cognitive aspects such as \textit{seeing things from an American point of view}.

Additionally, \isi{acculturation} was measured qualitatively through semi-struc\-tured interviews at the beginning and at the end of the semester, which revealed reasons for individual trajectories of \isi{sociocultural} and of \isi{psychological adaptation} during SA. The questions formulated were related to students’ \isi{acculturation} experiences following \citegen{Schumann1978} proposal of social and psychological \isi{acculturation} variables (c.f. \sectref{sec:sanchez:2.2}). Moreover, the semi-structured format of the interviews was advantageous for the elicitation of relevant topics that could explain \isi{acculturation} but were not included in Schumann’s proposal. More particularly, the following themes were pre-selected: educational background and \ili{English} experience in the home country; SA program goals and expectations; SA outcomes; academic and \isi{sociocultural} adjustment; overall well-being; \ili{English} use (interaction with \ili{English} speakers); \isi{pragmatic} awareness; and influence of instruction.

\subsection{Data collection}

This is a longitudinal study that employed a pre-test–post-test design. Data collection took place during the 2014 fall semester. For the pre-test, a day and time were established during the second week of the semester, during which participants were asked to complete the written instruments (the background questionnaire, the DCT, and the \isi{SCAS}) and to participate in the interviews. Completion of the written instruments took place during \isi{L2} \ili{English} classes and lasted for approximately 30 minutes, during which participants read and signed the consent form (5 minutes), completed the background questionnaire (5 minutes), the \isi{SCAS} (10 minutes), and the \isi{pragmatic} test (10 minutes). The interview sessions were held at the main researcher’s office, each lasting between 25 and 35 minutes, and they were recorded with the software Audacity. The post-test data collection sessions followed the same protocol used for the pre-test and took place during the week before the end of the semester.


\subsection{Data analysis}
\largerpage
The first type of data coded was quantitative information about L2\isi{pragmatic} competence. Pragmatic knowledge was operationalized in terms of \isi{appropriateness} of \isi{speech act production}, and it was evaluated by means of \isi{NSs}’ ratings in a holistic \isi{appropriateness} scale designed by \citet{Taguchi2011}. The instrument is a 5-point Likert scale that ranges from (1) very poor to (5) excellent. It assesses an answer to a DCT situation in terms of 3 aspects of \isi{pragmatic} competence: level of politeness, level of directness, and level of formality. \tabref{tab:sanchez:3} shows the rating scale used.


\begin{table}
\caption{Appropriateness rating scale developed by \citet[459]{Taguchi2011}}
\label{tab:sanchez:3}

\begin{tabularx}{\textwidth}{lQ}
\lsptoprule 
Excellent & Almost perfectly appropriate and effective in the level of directness, politeness and formality.\\
Good & Not perfect but adequately appropriate in the level of directness, politeness, and formality. Expressions are a little off from target-like, but pretty good.\\
Fair & Somewhat appropriate in the level of directness, politeness, and formality. Expressions are more direct or indirect than the situation requires.\\
Poor & Clearly inappropriate. Expressions sound almost rude or too demanding.\\
Very poor & Not sure if the target \isi{speech act} is performed.\\
\lspbottomrule
\end{tabularx}
\end{table}

Five \isi{NSs} were trained in the rating of \isi{pragmatic} \isi{appropriateness}. The training contained information about the purpose of the data collection, the coding criteria, some examples of previous studies that have used the \isi{appropriateness} scale of the study (\citealt{Taguchi2011,Taguchi2013}), and practice with data from the pilot study (\textit{N} = 21). Drawing from \citet{HudsonEtAl1995}, who proposed the use of an \isi{appropriateness} rating scale to assess \isi{pragmatic} production, the \isi{NSs} were instructed not to consider grammaticality. Inter-rater reliability was \textit{r} = 0.83. The disagreements (17\% of the data) were discussed and resolved during a meeting.

The second type of data coded was quantitative information about the students’ \isi{sociocultural} adaptation. Answers from the \isi{SCAS} were analyzed, revealing scores that ranged from 1, which indicated poor \isi{sociocultural} adaptation, to 5, meaning high levels of adaptation. 

Thirdly, the content of the semi-structured interviews was analyzed by eliciting different themes that allowed for the establishment of participant profiles based on the development of their \isi{sociocultural} and \isi{psychological adaptation}. Following \citegen{Schumann1986} proposal of \isi{acculturation} variables, comments in the interviews were coded into 5 main \isi{sociocultural} themes and 4 psychological ones. Sociocultural adaptation aspects included the \isi{integration strategy} adopted (which involved the development of social networks), enclosure, cohesiveness and size of the sojourning group, cultural congruence, and changes in attitude towards US culture. Intended length of residence and social dominance were not included in the analysis as there was homogeneity across the participants in these two aspects. Psychological adaptation aspects included \isi{culture shock}, \isi{language shock}, motivation, and \isi{ego permeability}. Additionally, I considered that the semi-structured interviews could reveal further factors that accounted for students’ adaptation, especially psychological factors \citep{Schumann1986}. Finally, \isi{acculturation} profiles of students were established by discerning whether \isi{sociocultural} and \isi{psychological adaptation} increased or decreased from the beginning to the end of the semester. To do so, I focused on answers given during the final interview, as well as on the change in answers given across time.  

Once the three types of data were coded, the next step was to analyze them to answer the two research questions of the study. Firstly, gains in \isi{pragmatic} production were calculated by means of a series of Wilcoxon Signed-Ranks tests. This non-parametric test, selected given the small sample size of participants, allowed for comparisons of average scores from the rating scales in the pre-test and the post-test. Secondly, gains in \isi{sociocultural} adaptation were calculated through an additional Wilcoxon Signed-Ranks test, and the relationship between \isi{sociocultural} adaptation and \isi{pragmatic} development was measured through the non-parametric Spearman rho correlation. To analyze the qualitative data, different individual trajectories of \isi{sociocultural} and \isi{psychological adaptation} were discerned, and they were compared against individual trajectories of \isi{pragmatic} learning. This allowed for the presentation and interpretation of case studies that illustrate patterns of associations between \isi{acculturation} and \isi{pragmatic} competence.


\section{Results}
\largerpage

\subsection{RQ1: Does a semester of study abroad afford gains in L2 pragmatic competence, in terms of speech act production?}

The first \isi{research question} of the study asked whether students improved their \isi{pragmatic} competence in the SA context, \isi{pragmatic} ability being operationalized in terms of \isi{appropriateness} of \isi{speech act production}, which was measured on a 5-point scale. To determine whether there were statistical differences between pre-test and post-test \isi{pragmatic} performance, Wilcoxon Signed-Ranks tests were conducted for \isi{speech act production} in high- and low-imposition situations, both including requests and refusals. Moreover, gains in overall \isi{speech act production} were calculated. \tabref{tab:sanchez:4} displays pre-test and post-test means (\textit{M}), standard deviations (\textit{SD}), and differences – which indicate gains – for each of the three aspects. Differences were considered significant at \textit{p} < 0.05.


\begin{table}
\caption{Descriptive statistics of production of speech acts}
\label{tab:sanchez:4}
\begin{tabularx}{\textwidth}{lrrrrrr}
\lsptoprule
& \multicolumn{2}{c}{\bfseries Pre-test} & \multicolumn{2}{c}{\bfseries Post-test} & \multicolumn{2}{c}{\bfseries Difference}\\
& \bfseries \textit{M} & \bfseries \textit{SD} & \bfseries \textit{M} & \bfseries \textit{SD} & \bfseries Score & \bfseries \%\\
\midrule
High-imposition situations & 3.12 & 0.74 & 3.56 & 0.79 & 0.44\,~ & 10.94\,~ \\
Low-imposition situations & 2.80 & 0.63 & 3.35 & 0.64 &  0.56* &  14.06*\\
\midrule
Overall production of speech acts & 2.96 & 0.47 & 3.45 & 0.55 &  0.50* &  12.50*\\
\lspbottomrule
\end{tabularx}
\footnotesize
*Significant at \textit{p} < 0.05
\end{table}

The statistical analysis revealed that participants improved their overall \isi{pragmatic} competence during the first semester of \isi{immersion} in the US (\textit{Z} = 2.31; \textit{p} = .021). More particularly, they significantly improved their \isi{appropriateness} in the production of requests and refusals in low-imposition situations (\textit{Z} = 2.41; \textit{p} = .016), although they did not experience statistically-significant \isi{pragmatic} gains in high-imposition situations (\textit{Z} = 1.69; \textit{p} = .09). To illustrate these findings, example (4) reflects positive gains in \isi{pragmatic} production in a low-imposition situation, and example (5) shows no gains in a high-imposition context. Both examples include answers in the pre- and post-tests to request situations by the same participant, Lisa, and they also show the agreed assessment of the responses by the raters in the \isi{appropriateness} rating scale.

\ea
{You are in class and you need to write something down, but you realize you forgot your pen at home. You tell the classmate sitting next to you:}\\
  \ea {Pre-test: \textit{I was wondering if you have a pen I could maybe borrow}}\\
  \textup{Evaluation}: (3) fair. Somewhat appropriate in the level of directness, politeness, and formality. Expressions are more direct or indirect than the situation requires.\\

  \ex{Post-test: \textit{Do you have a pen I could borrow? Please?}}\\
  \textup{Evaluation}: (5) excellent. Almost perfectly appropriate and effective in the level of directness, politeness and formality.\\
  \z
\z

\ea
{You need to ask a professor for an extension of a deadline for turning in a paper. At the end of a class session you tell him:} \\
  \ea {Pre-test: \textit{Sorry, could I have an extension of the deadline for the Biology paper?}}\\
  \textup{Evaluation}: (3) fair. Somewhat appropriate in the level of directness, politeness, and formality. Expressions are more direct or indirect than the situation requires.\\
  
  \ex {Post-test: \textit{Excuse me, can I have an extension of the deadline for the paper?}}\\
  \textup{Evaluation}: (3) fair. Somewhat appropriate in the level of directness, politeness, and formality. Expressions are more direct or indirect than the situation requires.\\
  \z
\z

As may be observed in example (4), Lisa improved her \isi{appropriateness} in requesting a pen from a friend (that is, a low-imposition situation) by using the conventional request strategy “Do you have.... I can borrow?”  which was more appropriate than the expression \textit{I was wondering} and the mitigator \textit{maybe}, which are more appropriately used in high-imposition situations. In contrast, example (5) shows that she did not improve her \isi{appropriateness} in requesting an extension of a deadline from a professor (that is, a high-imposition situation). Although she provided different answers in the pre- and post-tests, both answers were rated as \textit{fair}, as she could have used a more polite and indirect strategy.

The present findings in relation to \isi{research question} 1 have pointed out that the SA context is beneficial for the production of speech acts, particularly in low-imposition situations that involve conversations with friends. This suggests that during the first semester of \isi{immersion}, students were probably mostly exposed to informal situations and interactions with other students rather than to formal conversations with professors. These results are in line with previous studies that have revealed that the first months of \isi{immersion} in the SA context enhance the production of low-imposition speech acts to a greater extent than that of high-imposition speech acts such as requests, refusals, and opinions (\citealt{Taguchi2006}; \citeyear{Taguchi2011}; \citeyear{Taguchi2013}).


\subsection{RQ2: To what extent, if any, does students’ acculturation development influence gains in L2 pragmatic competence?}
\largerpage

The second \isi{research question} of the study asked to what extent, if any, students’ \isi{acculturation} development was related to the gains in \isi{pragmatic} competence reported above (see §5.1). Acculturation was quantitatively operationalized in terms of \isi{sociocultural} adaptation, while a qualitative analysis accounted for both \isi{sociocultural} and \isi{psychological adaptation}. 

To determine whether the participants experienced gains in their \isi{sociocultural} adaptation, a Wilcoxon Signed-Ranks test compared pre-test adaptation scores (\textit{M} = 3.80; \textit{SD} = 0.45; Min = 3.24; Max = 4.66) with post-test ones (\textit{M} = 4.04; \textit{SD} = 0.42; Min = 3.28; Max = 4.55). Results indicated a significant improvement (\textit{Z} = 1.778; \textit{p} = 0.075), and therefore confirmed that a semester of SA enhanced the students’ \isi{sociocultural} adaptation. 

  
To examine the relationship between the reported \isi{sociocultural} adaptation gains and the L2 \isi{pragmatic} gains, a Spearman rho correlation was conducted, with the significance level established at \textit{p} < 0.10. The analysis revealed a positive correlation (\textit{r} = 0.604, \textit{p} = 0.038). Therefore, we may hypothesize that adaptation to the SA context may have played a key role in learners’ improvement of their ability to produce speech acts. 

Next, post-hoc Spearman rho correlation tests between adaptation gains and gains in \isi{speech act production} in high-imposition and in low-imposition situations were calculated. The results revealed that \isi{sociocultural} adaptation was significantly related to \isi{speech act production} in high-imposition situations (\textit{r} = 0.527; \textit{p} = 0.079), but it was unrelated to \isi{speech act production} in low-imposition situations (\textit{r} = 0.424; \textit{p} = 0.169). This finding suggests that students who improved their \isi{sociocultural} adaptation during the semester were also likely to improve their \isi{pragmatic} ability in high-imposition formal situations that involve interacting with a professor. It could also imply that students who improved their \isi{pragmatic} ability in such situations were also likely to improve their adaptation to the new context. Nevertheless, gains in \isi{pragmatic} ability in low imposition situations such as interacting with friends were unrelated to \isi{sociocultural} adaptation. 

Next, the association between \isi{acculturation} and \isi{pragmatic} development was explored qualitatively. The participants’ answers in the interviews at the beginning and at the end of the semester were analyzed, and individual profiles in terms of \isi{sociocultural} and \isi{psychological adaptation} were established. \tabref{tab:sanchez:5} illustrates the 12 individual profiles by showing descriptive data about their \isi{pragmatic} gains (in high- and low-imposition situations) – expressed in percentages – their \isi{sociocultural} adaptation gains – expressed in gain scores from the analysis of \isi{SCAS} answers – and overall increase or decrease in \isi{sociocultural} and \isi{psychological adaptation}. The ordering of participants in the table is hierarchical, running from the student with the greatest positive gains in overall \isi{pragmatic} competence to the one with greatest negative gains.


\begin{table}
\caption{Descriptive information of the development of pragmatic competence and acculturation by 12 informants}
\label{tab:sanchez:5}
\fittable{
\small
\begin{tabular}{l rrrr ll}
\lsptoprule
	    & \multicolumn{4}{c}{\bfseries Pragmatic competence} & \multicolumn{2}{c}{\bfseries Acculturation}\\
	    & \bfseries Participant & \bfseries High imp%osition 
						    & \bfseries Low imp%osition
							      & \bfseries Overall & {\bfseries Sociocultural}  & {\bfseries Psychological} \\
\midrule	    
{David}     &   {2}         &      {1.75}     &     {1.9}     &   {0.97}      &      {Increase}  &    {Increase}\\
{Emma}      &   {1}         &      {1.75}     &     {1.4}     &   {0.89}      &      {Increase}  &    {Increase}\\
{Mike}      &   {1.75}      &      {0.5}      &     {1.2}     &   {--0.04}   &      {Decrease}  &    {Increase}\\
{Sean}      &   {0.5}       &      {1.25}     &     {0.8}     &   {0.48}      &      {Increase}  &    {Decrease}\\
{Lisa}      &   {0.5}       &      {0.5}      &     {0.5}     &   {0.6}       &      {Increase}  &    {Increase}\\
{Jeff}      &   {--0.25}   &      {1.25}     &     {0.5}     &   {0.41}      &      {Increase}  &    {Increase}\\
{William}   &   {0}         &      {0.75}     &     {0.4}     &   {--0.18}   &      {Decrease}  &    {Increase}\\
{Steven}    &   {0.75}      &      {0}        &     {0.4}     &   {--0.01}   &      {Decrease}  &    {Increase}\\
{Jason}     &   {0}         &      {--0.25}  &     {--0.1}  &   {--0.08}   &      {Decrease}  &    {Decrease}\\
{Ethan}     &   {--0.25}   &      {--0.25}  &     {--0.2}  &   {0.07}      &      {Increase}  &    {Decrease}\\
{Michelle}  &   {--0.5}    &      {0}        &     {--0.3}  &   {0.03}      &      {Increase}  &    {Decrease}\\
Mark        &   --0.25     &      --0.5     &     --0.4    &   --0.27     &      Decrease    &    Decrease  \\
\midrule 
average     & 0.44	 & 0.56 		& 0.5 		& 0.24   \\
\lspbottomrule
\end{tabular}
}
\end{table}

 
\tabref{tab:sanchez:5} shows that there were diverse individual trajectories of \isi{pragmatic} learning and \isi{acculturation}. Taking all the profiles into account, 3 patterns can be observed, which will guide the presentation of the qualitative findings in the following sections. Pattern 1 includes informants whose gains in \isi{pragmatic} competence – either positive or negative – corresponded with their gains in both overall \isi{sociocultural} and \isi{psychological adaptation}. Pattern 2 refers to participants whose \isi{pragmatic} gains only corresponded with their \isi{psychological adaptation} gains. Finally, pattern 3 includes one informant whose gains in \isi{speech act production} only corresponded with his gains in \isi{sociocultural} adaptation.


\subsubsection{Pattern 1: Interplay of sociocultural and psychological adaptation and pragmatic gains}

The first category includes informants who have shown either positive or negative gains in \isi{pragmatic} production and in both \isi{sociocultural} and psychological outcomes of \isi{acculturation}. David, Emma, Lisa and Jeff were gainers in this respect, while Jason, Mark and Michelle were non-gainers.

On the one hand, David, Jeff, Emma and Lisa all showed positive gains in \isi{pragmatic} competence, \isi{sociocultural} adaptation, and \isi{psychological adaptation}. Their \isi{sociocultural} adaptation was mainly determined by the successful \isi{integration strategy} adopted: that of assimilation of US \isi{sociocultural} values. They were the only participants that consciously tried to interact beyond their home-country cohesive group (in this case, Brazilians and Spaniards) and to assimilate host values. David and Lisa’s integration can be largely attributed to making close US friends and, in the case of David, finding a NS girlfriend. Jeff and Emma’s successful integration and therefore \isi{pragmatic} development were mainly due to their enrollment in clubs – a theatre club for Jeff, and a music band and volunteering program for Emma. At the same time, the four improved their \isi{psychological adaptation} thanks to social support from their home-country peers. In this sense, both David and Jeff expressed that living with their “Brazilian family” was what made the experience great. Similarly, Lisa expressed that although she tried to spend most of her time with her US roommates in order to integrate into the community, she also felt she had a \ili{Spanish} family, and indeed all of the \ili{Spanish} students developed a close relationship. In the case of Emma, her improvement in \isi{psychological adaptation} was primarily attributed to a reduction of \isi{language shock}, which was a consequence of her integration into the \isi{L2} community.

On the other hand, Jason, Mark and Michelle decreased their scores of \isi{pragmatic} ability, as well as their \isi{sociocultural} and \isi{psychological adaptation}. In the three cases, the students were not able to integrate into the \isi{L2} society, and instead preserved their own \isi{sociocultural} values during the stay. This unsuccessful integration may be due to several reasons: an increase in \isi{language shock} and a consequent change in personality in the case of Jason, academic pressure in the case of Mark, and \isi{ego permeability} (also a problem related to personality) in the case of Michelle. Jason developed a wide network of Brazilian friends and did not gain confidence to interact with Americans. In the post-test interview, he expressed that his anxiety to speak in \ili{English} had increased and that he was disappointed because he had expected to improve his speaking ability by coming to the US. Moreover, it seems that his \isi{language shock} made him shy when using \ili{English}, as he claimed to be extroverted and social with his Brazilian friends but could not interact with US students. Similarly, Michelle was aware that she was a shy person, and according to her, her introverted personality prevented her from interacting with \isi{NSs} and from learning about their culture. As for Mark, his strong motivation to integrate into the society and to practice his \ili{English} was evident: he enrolled in university clubs and also reached out to \isi{NSs}. Nevertheless, he regretted feeling a lot of pressure to pass a TOEFL test at the end of the semester that would enable him to continue in the SA program. As a consequence, both Michelle and Mark also increased their \isi{language shock} and at the end of the semester and reported being scared or ashamed of using their \ili{English} at times.


\subsubsection{Pattern 2: Interplay of psychological adaptation and pragmatic gains}
 
The second case involves participants whose \isi{pragmatic} gains corresponded with their \isi{psychological adaptation} gains, but not with their \isi{sociocultural} ones. Wil\-liam, Mike, and Steven experienced positive gains in \isi{speech act production} and in \isi{psychological adaptation}, while Ethan showed negative \isi{pragmatic} development and a decrease in his \isi{psychological adaptation}.

\newpage 
William and Mike experienced similar \isi{sociocultural} and \isi{psychological adaptation} paths. They reported having a phenomenal semester thanks to the Brazilian friends they made. Therefore, social support from their home-country peers seemed to have enhanced their well-being and their \isi{psychological adaptation}. Nevertheless, limiting their contact to Brazilians made their \isi{sociocultural} adaptation decrease. This situation was more striking in William’s case, who openly admitted not making contacts outside his Brazilian peer group. William, however, claimed that he learned a great deal of \ili{English} since sometimes the Brazilians spoke in \ili{English} among themselves, so he attributed his \isi{language improvement} to the meta-talk resulting from home-country peers correcting each other. Mike did integrate to some extent into the \isi{L2} community, and apart from Brazilian colleagues, he made friends with mainly other international students. In the case of Steven, his \isi{psychological adaptation} increased during the semester from making friends of different nationalities, and his attitude towards the US improved as a result of this positive experience. Nevertheless, he did not experience \isi{immersion} in US culture or make any NS friends. Steven’s increased well-being could explain his moderate but positive \isi{pragmatic} gains. Additionally, the fact that he was mainly concerned with improving his academic \ili{English} (a reflection of his \isi{instrumental motivation}) could account for his positive speech act production development in high-imposition situations. 

  
Unlike William, Mike and Steven, Ethan did show gains in his \isi{sociocultural} adaptation, mainly due to a progression towards a more positive attitude regarding US culture. Nevertheless, he experienced negative gains in his \isi{pragmatic} competence as well as in his \isi{psychological adaptation}. His decrease in \isi{psychological adaptation} was mainly due to his lack of \isi{ego permeability}. Ethan described himself as an introverted person whose preferred plan for a Saturday evening during the stay abroad was playing video games with a \ili{Spanish} peer, who became his best friend. He also admitted not trying very hard to integrate with Americans since his main motivation in the program was to improve academically, not socially or personally.


\subsubsection{Pattern 3: Interplay of sociocultural adaptation and pragmatic gains}

The third category includes one student whose \isi{pragmatic} gains corresponded to his \isi{sociocultural} adaptation development, but not with his psychological one. This is the case of Sean, whose \isi{psychological adaptation} decreased over the semester because his \isi{culture shock} increased. His inability to cope with some cultural differences – particularly with the US custom of keeping dogs indoors and not removing shoes inside the house – led him to have arguments with his American roommates and change his living arrangements. Sean finally felt well-adapted to the setting when, by the end of the semester, he changed from having two US roommates to living with two international students, one from Saudi Arabia and one from Thailand. He particularly felt his \ili{English} improved more when sharing accommodation with international students since they interacted frequently. Sean’s \isi{sociocultural} adaptation improved during the semester abroad, which was mainly due to the fact that he felt more integrated into the \isi{L2} community once he had made real friends. Even if his friends were of other nationalities, going out with them gave him confidence to interact more with \isi{NSs}.

In summary, a qualitative exploration of individual trajectories seems to indicate that gains in \isi{appropriateness} of \isi{speech act production} were somewhat related to overall \isi{acculturation} gains. More particularly, the analysis highlighted the key role of the social variables of integration and of academic pressure, as well as the importance of the affective variables of social support from home-country peers and \isi{language shock} in shaping the process of \isi{acculturation}. It may also be hypothesized that \isi{psychological adaptation} might lead to \isi{pragmatic} gains to a higher extent than \isi{sociocultural} adaptation, as only in one case (Sean) did \isi{sociocultural} gains correspond with \isi{pragmatic} gains, as opposed to the four cases in which only \isi{psychological adaptation} was associated with \isi{pragmatic} development (William, Mike, Steven and Ethan).


\section{Discussion}
 
The current study investigated whether \isi{acculturation} was related to the development of L2 \isi{pragmatic} competence in the SA context. More particularly, it sought to discern (1) whether students developed their \isi{appropriateness} of \isi{pragmatic} production during a semester of study in the US, and (2) whether and how their \isi{sociocultural} and \isi{psychological adaptation} were related to the reported \isi{pragmatic} gains. The objectives of the study drew on \citegen{Schumann1978} proposal that the degree to which an individual acculturates socially and affectively to the \isi{L2} setting would determine the extent to which he/she learns the \isi{L2}.

Firstly, the results corroborate previous longitudinal ILP studies that have reported that, despite the advantage of the SA context for \isi{pragmatic} development, the process of learning \isi{pragmatic} competence is variable and non-linear, as it is determined by different factors (e.g. \citealt{Barron2003}; \citealt{Félix-Brasdefer2004}; \citealt{TaguchiRoever2017}). Overall, the findings revealed that during the first 4 months of \isi{immersion} in the US, participants improved their ability to formulate requests and refusals, as measured on a DCT. Nevertheless, this improvement was influenced by the type of situation, since learners showed higher gains in \isi{appropriateness} of \isi{speech act production} in low-imposition situations involving friends than in high-imposition ones entailing conversations with professors. This finding is, indeed, in line with a series of longitudinal investigations by \cite{Taguchi2011, Taguchi2013} reporting that the development of high-imposition \ili{English} speech acts (requests, refusals, and opinions) seems to take place at later stages of \isi{L2} \isi{immersion}. It is hypothesized that low-imposition scenarios presented in this study (see \tabref{tab:sanchez:2}) were encountered on campus more frequently than the high-imposition ones. This finding thus highlights the particularity of SA as an optimal setting for \isi{L2} learning given the opportunities it provides for interaction outside of class in different situations and for exposure to contextualized and authentic input \citep{Taguchi2015contextually}.

 
Moreover, the study has revealed that L2 \isi{pragmatic} development was related in various ways to students’ \isi{acculturation} experiences. Therefore, it provides support for \citeauthor{Schumann1978}'s (\citeyear*{Schumann1978,Schumann1986}) Acculturation model of \isi{L2} \isi{acquisition} and corroborates findings from previous studies that have provided empirical evidence for the relationship between \isi{acculturation} and L2\isi{pragmatic} \isi{acquisition} (\citealt{Schmidt1983}; \citealt{DörnyeiEtAl2004}). A quantitative analysis showed that \isi{sociocultural} adaptation and \isi{pragmatic} gains were significantly associated. Moreover, a qualitative exploration of individual trajectories of \isi{pragmatic} learning and of \isi{acculturation} showed that different \isi{sociocultural} and psychological factors contributed to adaptation to the \isi{L2} setting and to subsequent learning of how to use the \isi{L2} appropriately as a function of interlocutor and situation. On the one hand, \isi{acculturation} was determined by social variables that mainly included integration and academic pressure. On the other hand, it was influenced by affective factors, primarily social support from home-country peers and \isi{language shock}. The most successful students in \isi{pragmatic} learning were those who were able to integrate into the \isi{L2} community and those whose \isi{psychological adaptation} increased thanks to the support from their home-country peers. In contrast, unsuccessful students were not able to integrate into US society mainly because of strong \isi{language shock} and high academic pressure. 

Although academic pressure and social support from home-country peers were not a primary focus of this investigation, they were found to be related to \isi{acculturation} in the US. This finding has two main implications. Firstly, it supports \citegen{Schumann1978} assertion that \isi{acculturation}, rather than being a direct cause of \isi{L2} \isi{acquisition}, is one of the main factors enhancing \isi{L2} learning. Secondly, it highlights the need to revise \citegen{Schumann1978} framework so as to account for language learning in the current era of globalization, which has resulted in a dramatic increase of student mobility worldwide (see \citealt{MitchellEtAl2015,MitchellEtAl2017}). For instance, nowadays it is common to have large groups of students from the same nationality, and even from the same home university, at the same SA site, which makes social support from home-country peers an inevitable element to take into account when investigating the SA setting. 

All in all, the findings from this study provide new insights into the investigation of SA program outcomes by reporting the relationship between L2 \isi{pragmatic} development and sociocultural and \isi{psychological adaptation}. Therefore, the study has addressed the recent call for the need to investigate cultural and \isi{linguistic} aspects of SA so as to have a more comprehensive understanding of the SA experience \citep{Taguchi2015crosscultural}. To that end, future studies could draw from \citegen{Schumann1978} framework of Acculturation.


\section{Conclusion}

This investigation focused on the interplay of \isi{acculturation} and the \isi{acquisition} of L2 \isi{pragmatic} competence in the SA context. More particularly, it discussed ways in which students’ \isi{sociocultural} and \isi{psychological adaptation} during the first 4 months of \isi{immersion} could be related to their ability to formulate requests and refusals appropriate to the given situation.

  
The main limitations of the study concern the relatively small sample size, and the merely qualitative assessment of \isi{psychological adaptation}. A qualitative analysis of 12 case studies was, however, the most suitable methodology for the purpose of the study, since, as different scholars have noted (e.g. \citealt{DörnyeiEtAl2004}; \citealt{Taguchi2011}), it allows for an in-depth exploration of the interplay between contextual factors and learners’ individual differences. Nevertheless, future research should include a higher number of participants, as well as administer quantitative measures of \isi{psychological adaptation} or overall \isi{acculturation}.

Ultimately, the study makes a notable contribution to the field of ILP by highlighting the relevance of \isi{acculturation}. Moreover, the findings have important implications for the design of mobility programs, as they highlight the need to maximize students’ \isi{immersion} experiences during SA programs, both at the social and at the affective level, to enhance their ability to use the \isi{L2} appropriately in the new \isi{sociocultural} setting. 

\section*{Acknowledgments}

As a member of the LAELA (Lingüística Aplicada a l’Ensenyament de la
Llengua Anglesa) research group at Universitat Jaume I (Castellón, Spain),
I would like to acknowledge that this study is part of a research project
funded by (a) the Spanish Ministerio de Economia y Competitividad
(FFI2016-78584-P), (b) the Universitat Jaume I (P1·1B2015-20), and (c)
Projectes d’Innovació Educativa de la Unitat de Suport Educatiu 3457/17.


\sloppy
\printbibliography[heading=subbibliography,notkeyword=this] 
\end{document}